%\documentclass[12pt]{article}

\questionheader{ex:s1.1}

%%%%%%%%%%%%%%%%%%%
%\subsection*{Derivatives, Velocity, Etc.}
%%%%%%%%%%%%%%%%%%%


%%%%%%%%%%%%%%%%%%
\subsection*{\Conceptual}
%%%%%%%%%%%%%%%%%%

%%%%%%%%%%%%%%%%%%%%%%%%%%%
\Instructions{Questions~\ref{prob_s1.0first} through \ref{prob_s1.0last} provide practice with curve parametrization. Being comfortable with the algebra and interpretation of these descriptions  are essential ingredients in working effectively
               with parametrizations.}

%%%%%%%%%%%%%%%%%%%%%%%%%%%%%%%
\begin{question}\label{prob_s1.0first}
Find the specified parametrization of the first quadrant part
of the circle $x^2+y^2=a^2$.
\begin{enumerate}[(a)]
\item 
  In terms of the $y$ coordinate.
\item
  In terms of the angle between the tangent line and the 
  positive $x$-axis.
\item
  In terms of the arc length from $(0,a)$.
\end{enumerate}
\end{question}

\begin{hint} 
Draw sketches. Don't forget the range that the parameter runs over.
\end{hint}

\begin{answer} 
(a) $\vr(y)=\sqrt{a^2-y^2}\,\hi+ y\,\hj$, $0\le y\le a$

(b) $\big(x(\phi),y(\phi)\big)
       =\big(a\sin \phi ,-a\cos \phi \big)$,
   $\frac{\pi}{2}\le\phi\le\pi$

(c) $\big(x(s),y(s)\big)
    =\big(a\cos(\tfrac{\pi}{2}-\frac{s}{a}),
           a\sin(\tfrac{\pi}{2}-\tfrac{s}{a})\big)$,
   $0\le s\le\tfrac{\pi}{2}a$
\end{answer}

\begin{solution} 
(a) 
Since, on the specified part of the  circle, 
$x=\sqrt{a^2-y^2}$ and  $y$ runs from $0$ to $a$, 
the parametrization is
$\vr(y)=\sqrt{a^2-y^2}\,\hi+ y\,\hj$, $0\le y\le a$.

(b) Let $\theta$ be the angle between 
\begin{itemize}\itemsep1pt \parskip0pt \parsep0pt
\item the radius vector from the origin to the point 
$(a\cos\theta,a\sin\theta)$  on the circle and 
\item
the positive $x$-axis. 
\end{itemize}
The tangent line to the circle at $(a\cos\theta,a\sin\theta)$ 
is perpendicular to the radius vector and so makes angle $\phi=\frac{\pi}{2}+\theta$ with the positive $x$ axis.
(See the figure on the left below.)
As $\theta =\phi-\frac{\pi}{2}$, the desired parametrization is
\begin{equation*}
\big(x(\phi),y(\phi)\big)
=\big(a\cos(\phi-\tfrac{\pi}{2}),a\sin(\phi-\tfrac{\pi}{2})\big)
=\big(a\sin \phi ,-a\cos \phi \big),\ 
  \tfrac{\pi}{2}\le\phi\le\pi
\end{equation*}

\begin{center}
       \includegraphics{fig/parCirclePhi.pdf}\quad
       \includegraphics{fig/parCircleS.pdf}
\end{center}


(c) Let $\theta$ be the angle between 
\begin{itemize}\itemsep1pt \parskip0pt \parsep0pt
\item the radius vector from the origin to the point 
$(a\cos\theta,a\sin\theta)$  on the circle and 
\item
the positive $x$-axis. 
\end{itemize} 
The arc from $(0,a)$ to $(a\cos\theta,a\sin\theta)$ 
subtends an angle $\frac{\pi}{2}-\theta$ and so
has length $s=a\big(\frac{\pi}{2}-\theta\big)$. (See the figure
on the right above.) Thus $\theta=\frac{\pi}{2}-\frac{s}{a}$
and the desired parametrization is
\begin{equation*}
\big(x(s),y(s)\big)
=\left(a\cos\left(\frac{\pi}{2}-\frac{s}{a}\right)\,,\,
           a\sin\left(\frac{\pi}{2}-\frac{s}{a}\right)\right)
,\ 0\le s\le\frac{\pi}{2}a
\end{equation*}
\end{solution}





%%%%%%%%%%%%%%%%%%%
\begin{question}
Consider the following time-parametrized curve:
\[\vr(t)=\left( \cos\left(\frac{\pi}{4}t\right)~,~(t-5)^2\right)\]

List the three points $(-1/\sqrt{2},0)$, $(1,25)$, and $(0,25)$ in chronological order.
\end{question}
\begin{hint}
Find the value of $t$ at which the three points occur on the curve.
\end{hint}
\begin{answer}
$(1,25)$, $(-1/\sqrt2,0)$, $(0,25)$.
\end{answer}
\begin{solution}
We can find the time at which the curve hits a given point by considering the two equations that arise from the two coordinates. For the $y$-coordinate to be 0, we must have $(t-5)^2=0$, i.e. $t=5$. So, the point $(-1/\sqrt{2},0)$ happens when $t=5$.

Similarly, for the $y$-coordinate to be $25$,  we need $(t-5)^2=25$, so $(t-5)=\pm 5$.
When $t=0$, the curve hits $(1, 25)$; when $t=10$, the curve hits $(0,25)$.

So, in order, the curve passes through the points $(1,25)$, $(-1/\sqrt2,0)$, and $(0,25)$.
\end{solution}
%%%%%%%%%%%%%%%%%%%
\begin{question}
At what points in the $xy$-plane does the curve $(\sin t, t^2)$ cross itself? What is the difference in $t$ between the first time the curve crosses through a point, and the last?
\end{question}
\begin{hint}
The curve ``crosses itself" when $(\sin t,t^2)$ gives the same coordinate for different values of $t$. When these crossings occur will depend on which crossing you're referring to, so your answers should all depend on $t$.
\end{hint}
\begin{answer}
The curve crosses itself at all points $(0,(\pi n)^2)$ where $n$ is an integer. It passes such a point twice, $2\pi n$ time units apart.
\end{answer}
\begin{solution}
The curve ``crosses itself" when the same coordinates occur for different values of $t$, say $t_1$ and $t_2$. So, we want to know when $\sin t_1=\sin t_2$ and also $t_1^2=t_2^2$. Since $t_1$ and $t_2$ should be different, the second equation tells us $t_2=-t_1$. Then the first equation tells us $\sin t_1=\sin t_2=\sin(-t_1)=-\sin t_1$. That is, $\sin t_1 = -\sin t_1$, so $\sin t_1=0$. That happens whenever $t_1=\pi n$ for an integer $n$.

So, the points at which the curve crosses itself are those points $(0,(\pi n)^2)$ where $n$ is an integer. It passes such a point at times $t=\pi n $ and $t=-\pi n$. So, the curve hits this point $2\pi n$ time units apart.
\end{solution}
%%%%%%%%%%%%%%%%%%%%%%%%%%%%%%%%%%%%%%
\begin{question}\label{prob_s1.1_cycloid}

\begin{center}
\begin{tikzpicture}
\YEaaxis{1}{13}{.5}{3}
\YExcoord{1}{a}
\YEycoord{1}{a}
\draw[thick] (1,2) node[red,vertex, label=above:{\textcolor{red}{$P$}}]{} arc (90:450:1cm);
\foreach \x in {0,1,2,3,4}{
	\MULTIPLY{\x}{.2}{\y}
	\SUBTRACT{1}{\y}{\o}
	\MULTIPLY{\x}{2.25}{\z}
	\MULTIPLY{\z}{57.3}{\zrad}
	\draw[red,opacity=\o] ({1+\z+sin(\zrad)},{1+cos(\zrad)}) node[vertex](P\x){};
	\draw[opacity=\o] ({1+\z},1) node[shape=circle, draw, minimum size=2cm]{}--(P\x);
		}
%\draw[thick, opacity=0.75] (1,2) node[red,vertex, label=above:{\textcolor{red}{$P$}}]{} arc (90:450:1cm);
\draw[red] plot[smooth,domain=0:13]({1+\x+sin(\x r)},{1+cos(\x r)});
\end{tikzpicture}
\end{center}

A circle of radius $a$ rolls along the $x$-axis in the positive direction, starting with its centre at $(a,a)$. In that position, we mark the topmost point on the circle $P$. As the circle moves, $P$ moves with it. Let $\theta$ be  the angle the circle has rolled --- see the diagram below.
\begin{enumerate}[(a)]
\item Give the position of the centre of the circle as a function of $\theta$.
\item Give the position of $P$ as a function of $\theta$.
\end{enumerate}
\begin{center}
\begin{tikzpicture}
\draw[thick] (0,0) node[shape=circle, minimum size=4cm, draw]{};
\draw (1.4,1.4)--(0,0)--(0,2);
\draw[red] (1.4,1.4) node[vertex, label= right:$P$](P){};
\draw[] (0,1) arc (90:45:1cm) node[midway, above]{$\theta$};
\draw[->] (0,2.5) arc(90:45:2.5cm);
\end{tikzpicture}
\end{center}

\end{question}
\begin{hint}
For part (b), find the position of $P$ relative to the centre of the circle. Then combine your answer with part (a).
\end{hint}
\begin{answer}
(a) $(a+a\theta,a)$\qquad
(b)$(a+a\theta+a\sin\theta,a+a\cos\theta)$
\end{answer}
\begin{solution} (a)
Pretend that the circle is a spool of thread. As the 
           circle rolls, it dispenses the thread along the ground.
           When the circle rolls $\theta$ radians, it dispenses the
           arc length $\theta a$ of thread and the circle advances
           a distance $\theta a$. So the centre of the circle has 
           moved $\theta a$ units to the right from its starting point, $x=a$. The centre of the circle always has $y$-coordinate $a$. So, after rolling $\theta$ radians, the centre of the circle is at position $\vc(\theta)=(a+a\theta,a)$.

(b) 
Now, let's consider the position of $P$ on the circle, after the circle has rolled $\theta$ radians. 

\begin{center}
\begin{tikzpicture}
\draw[thick] (0,0) node[shape=circle, minimum size=8cm, draw]{};
\draw (2.8,2.8)--(0,0)--(0,4);
\draw[red] (2.8,2.8) node[vertex, label=above right:$P$](P){};
\draw[] (0,1) arc (90:45:1cm) node[midway, above]{$\theta$};
\draw[dashed] (P)--(0,2.8) node[midway, above]{$a\sin\theta$};
\draw[decorate, decoration={brace, amplitude=10pt}](-.25,0)--(-.25,2.8) node[midway,rotate=90,yshift=7.5mm]{$a\cos\theta$};
\draw[->] (0,5) arc(90:45:5cm);
\end{tikzpicture}
\end{center}
From the diagram, we see that $P$ is $a\cos \theta$ units above the centre of the circle, and $a\sin \theta$ units to the right of it. So, the position of $P$ is $(a+a\theta+a\sin\theta,a+a\cos\theta)$.

Remark: this type of curve is known as a \emph{cycloid}.
\end{solution}
%%%%%%%%%%%%%%%%%%%
\begin{question}\label{prob_s1.0last}
The curve $C$ is defined to be the intersection of the ellipsoid
\[x^2-\frac{1}{4}y^2+3z^2=1\]
and the plane
\[x+y+z=0.\]
When $y$ is very close to 0, and $z$ is negative, find an expression giving $z$ in terms of $y$.
\end{question}
\begin{hint}
We aren't concerned with $x$, so we can eliminate it by solving one equation for $x$ as a function
        of $y$ and $z$ and plugging the result into the other equation.
        \end{hint}
\begin{answer}
$z=-\frac12\sqrt{1-\frac{y^2}{2}}-\frac{y}{4}$
\end{answer}
\begin{solution}
We aren't concerned with $x$, so we can eliminate it by solving for it in one equation, and plugging that into the other. Since $C$ lies on the plane, $x=-y-z$, so:
\begin{align*}
1&=x^2-\frac{1}{4}y^2+3z^2=(-y-z)^2-\frac14y^2+3z^2\\
&=\frac{3}{4}y^2+4z^2+2yz
\intertext{Completing the square,}
1&=\frac{1}{2}y^2+\left(2z+\frac{y}{2}\right)^2\\
1-\frac{y^2}{2}&=\left(2z+\frac{y}{2}\right)^2
\intertext{Since $y$ is small, the left hand is close to $1$
         and the right hand side is close to $(2z)^2$.
         So $(2z^2)\approx 1$. Since $z$ is negative,
         $z\approx -\frac{1}{2}$ and  $2z+\frac{y}{2}<0$. Also, $1-\frac{y^2}{2}$ is positive, so it has a real square root.}
-\sqrt{1-\frac{y^2}{2}}&=2z+\frac{y}{2}\\
-\frac12\sqrt{1-\frac{y^2}{2}}-\frac{y}{4}&=z
\end{align*}

\end{solution}
%%%%%%%%%%%%%%%%%%%
\begin{question}
A particle traces out a curve in space, so that its position at time $t$ is \[\vr(t)=e^{-t}\,\hi+\frac{1}{t}\,\hj+(t-1)^2(t-3)^2\,\hk\] for $t > 0$. 

Let the positive $z$ axis point vertically upwards, as usual. When is the particle moving upwards, and when is it moving downwards? Is it moving faster at time $t=1$ or at time $t=3$?
\end{question}
\begin{hint}
To determine whether the particle is rising or falling, we only need to consider its $z$-coordinate. 
\end{hint}
\begin{answer}
The particle is moving upwards from $t=1$ to $t=2$, and from $t=3$ onwards.  The particle is moving downwards from $t=0$ to $t=1$, and from $t=2$ to $t=3$.

The particle is moving faster when $t=1$ than when $t=3$.
\end{answer}
\begin{solution}
To determine whether the particle is rising or falling, we only need to consider its $z$-coordinate: $z(t)=(t-1)^2(t-3)^2$. Its derivative with respect to time is $z'(t)=4(t-1)(t-2)(t-3)$. This is positive when $1<t<2$ and when $3<t$, so the particle is increasing on $(1,2) \cup (3,\infty)$ and decreasing on $(0,1) \cup (2,3)$.

If $\vr(t)$ is the position of the particle at time $t$, then its speed is $|\vr'(t)|$. We differentiate:
\[\vr'(t)=-e^{-t}\,\hi-\frac{1}{t^2}\,\hj+4(t-1)(t-2)(t-3)\hk\]
So, $\vr(1)=-\frac{1}{e}\,\hi-1\,\hj$ and $\vr(3)=-\frac{1}{e^3}\,\hi-\frac{1}{9}\,\hj$. The absolute value of every component of $\vr(1)$ is greater than or equal to that of the corresponding component of $\vr(3)$, so $|\vr(1)|>|\vr(3)|$. That is, the particle is moving more swiftly at $t=1$ than at $t=3$.

Note: We could also compute the sizes of both vectors directly: $|\vr'(1)|=\sqrt{\left(\frac{1}{e}\right)^2+(-1)^2}$, and $|\vr'(3)|=\sqrt{\left(\frac{1}{e^3}\right)^2+\left(-\frac{1}{9}\right)^2}$.
\end{solution}
%%%%%%%%%%%%%%%%%%%
\begin{question}
Below is the graph of the parametrized function $\vr(t)$. Let $s(t)$ be the arclength along the curve from $\vr(0)$ to $\vr(t)$.

\begin{center}
\begin{tikzpicture}
\draw[thick] plot[domain=-4.5:1]({4-\x*\x/4},\x);
\draw (4-1/16,-.5) node[vertex, label=right: $\vr(t+h)$](th){};
\draw (3,-2) node[vertex, label=right: $\vr(t)$](t){};
\draw (0,-4) node[vertex, label=below right: $\vr(0)$](zero){};
%\draw[|-|, red,yshift=10mm, xshift=-10mm] plot[domain=-4:-2]({4-\x*\x/4},\x);
%\draw[|-|, red,yshift=7.5mm, xshift=-7.5mm] plot[domain=-4:-.5]({4-\x*\x/4},\x);
%\draw[|-|, red,yshift=5mm, xshift=-5mm] plot[domain=-2:-.5]({4-\x*\x/4},\x);
%\draw[very thick, red, ->] (t)--(th);
\end{tikzpicture}
\end{center}
Indicate on the graph $s(t+h)-s(t)$ and $\vr(t+h)-\vr(t)$. Are the quantities scalars or vectors?

%Label $s(t)$, $s(t+h)$, $s(t+h)-s(t)$, and $\vr(t+h)-\vr(t)$. Which are scalars, and which are vectors?

%Lemma 1.1.3: draw a diagram, have students label $h$, $s(t+h)-s(t)$ etc.; which are constants, and which are vectors?
\end{question}
\begin{hint}
This is the setup from Lemma~\eref{CLP317}{lem:CVtanArclen} in the CLP-4. 
The two quantities you're labelling are related, but different.
\end{hint}
\begin{answer}

\begin{center}
\begin{tikzpicture}
\draw[thick] plot[domain=-4.5:1]({4-\x*\x/4},\x);
\draw (4-1/16,-.5) node[vertex, label=right: $\vr(t+h)$](th){};
\draw (3,-2) node[vertex, label=right: $\vr(t)$](t){};
\draw (0,-4) node[vertex, label=below right: $\vr(0)$](zero){};
%\draw[|-|, red,yshift=10mm, xshift=-10mm] plot[domain=-4:-2]({4-\x*\x/4},\x);
%\draw[|-|, red,yshift=7.5mm, xshift=-7.5mm] plot[domain=-4:-.5]({4-\x*\x/4},\x);
\draw[|-|, blue,yshift=-5mm, xshift=10mm] plot[domain=-2:-.5]({4-\x*\x/4},\x);
\draw[very thick, red, ->] (t)--(th);% node[midway, left]{$\vr(t+h)-\vr(t)$};
\end{tikzpicture}
\end{center}
The red vector is $\vr(t+h)-\vr(t)$. The arclength of the segment indicated by the blue line is the (scalar) $s(t+h)-s(t)$.

Remark: as $h$ approaches 0, the curve (if it's differentiable at $t$) starts to resemble a straight line, with the length of the vector $\vr(t+h)-\vr(t)$ approaching the scalar $s(t+h)-s(t)$. This step is crucial to understanding Lemma~\eref{CLP317}{lem:CVtanArclen} in the CLP-4 text. 
\end{answer}
\begin{solution}

\begin{center}
\begin{tikzpicture}
\draw[thick] plot[domain=-4.5:1]({4-\x*\x/4},\x);
\draw (4-1/16,-.5) node[vertex, label=right: $\vr(t+h)$](th){};
\draw (3,-2) node[vertex, label=right: $\vr(t)$](t){};
\draw (0,-4) node[vertex, label=below right: $\vr(0)$](zero){};
%\draw[|-|, red,yshift=10mm, xshift=-10mm] plot[domain=-4:-2]({4-\x*\x/4},\x);
%\draw[|-|, red,yshift=7.5mm, xshift=-7.5mm] plot[domain=-4:-.5]({4-\x*\x/4},\x);
\draw[|-|, blue,yshift=-5mm, xshift=10mm] plot[domain=-2:-.5]({4-\x*\x/4},\x);
\draw[very thick, red, ->] (t)--(th);% node[midway, left]{$\vr(t+h)-\vr(t)$};
\end{tikzpicture}
\end{center}
The red vector is $\vr(t+h)-\vr(t)$. The arclength of the segment indicated by the blue line is the (scalar) $s(t+h)-s(t)$.

Remark: as $h$ approaches 0, the curve (if it's differentiable at $t$) starts to resemble a straight line, with the length of the vector $\vr(t+h)-\vr(t)$ approaching the scalar $s(t+h)-s(t)$. This step is crucial to understanding Lemma~\eref{CLP317}{lem:CVtanArclen}  in the CLP-4 text. 

\end{solution}
%%%%%%%%%%%%%%%%%%%
%%%%%%%%%%%%%%%%%%%
\begin{question}
What is the relationship between velocity and speed in a vector-valued function of time?
\end{question}
\begin{hint} See the note just before Example~\eref{CLP317}{eg:paramCircleTan}.
\end{hint}
\begin{answer}
Velocity is a vector-valued quantity, so it has both a magnitude and a direction. Speed is a scalar --- the magnitude of the velocity. It does not include a direction.
\end{answer}
\begin{solution}
Velocity is a vector-valued quantity, so it has both a magnitude and a direction. Speed is a scalar --- the magnitude of the velocity. It does not include a direction.
\end{solution}
%%%%%%%%%%%%%%%%%%%

%%%%%%%%%%%%%%%%%%%
\begin{question}[M317 2005D] %1
Let $\vr(t)$ be a vector valued function. Let $\vr'$, $\vr''$ , and $\vr'''$ 
denote $\diff{\vr}{t}$, $\difftwo{\vr}{t}$ and 
$\frac{\mathrm{d^3}\hfil\vr\hfil}{\mathrm{d}{t}^3}$, respectively.
Express
\begin{equation*}
\diff{\hfill}{t}\big[ (\vr \times \vr')\cdot\vr'' \big]
\end{equation*}
in terms of $\vr$, $\vr'$ , $\vr''$ , and $\vr'''$. 
Select the correct answer.
\begin{enumerate}[(a)]
\item\ \  $(\vr'\times\vr'' )\cdot\vr'''$
\item\ \  $(\vr'\times\vr'' )\cdot\vr + (\vr\times\vr' )\cdot\vr'''$
\item\ \  $(\vr\times\vr' )\cdot\vr'''$
\item\ \  $0$
\item\ \  None of the above.
\end{enumerate}
\end{question}

\begin{hint} 
To simplify your answer, remember: the cross product of $\va$ and $\vb$ is a vector orthogonal to both $\va$ and $\vb$; the cross product of a vector with itself is zero; and two orthogonal vectors have dot product 0.
\end{hint}

\begin{answer} 
(c)
\end{answer}

\begin{solution}
By the product rule
\begin{align*}
\diff{\hfill}{t}\big[ (\vr \times \vr')\cdot\vr'' \big]
&= (\vr' \times \vr')\cdot\vr''
  +(\vr \times \vr'')\cdot\vr''
  +(\vr \times \vr')\cdot\vr'''
\end{align*}
The first term vanishes because $\vr'\times\vr'=\vZero$.
The second term vanishes because $\vr \times \vr''$ is perpendicular to
$\vr''$. So
\begin{equation*}
\diff{\hfill}{t}\big[ (\vr \times \vr')\cdot\vr'' \big]
= (\vr \times \vr')\cdot\vr'''
\end{equation*}
which is (c).
\end{solution}

%%%%%%%%%%%%%%%%%%%
\begin{question}
Show that, if the position and velocity vectors of a moving 
particle are always perpendicular, then the path of the particle lies on
a sphere.
\end{question}

\begin{hint} 
Evaluate $\diff{}{t} |\vr(t)|^2$.
\end{hint}

\begin{answer}
See the solution.
\end{answer}

\begin{solution}
we are told that $\vr(t)\perp\vr'(t)$, so that $\vr(t)\cdot\vr'(t)=0$, 
for all $t$. Consequently
$$
\diff{}{t}|\vr(t)|^2
=\diff{}{t}\big[\vr(t)\cdot\vr(t)\big]
=2\vr(t)\cdot\vr'(t)=0
$$
So $|\vr(t)|^2$ is a constant, say $A$, independent of time and $\vr(t)$
always lies on the sphere of radius $\sqrt{A}$ centred on the origin.
\end{solution}

%%%%%%%%%%%%%%%%%%
\subsection*{\Procedural}
%%%%%%%%%%%%%%%%%%

%%%%%%%%%%%%%%%%%%%%%%%%%%%
\begin{question}[M317 2005D] %3
Find the speed of a particle with the given position function
\begin{equation*}
\vr(t) = 5 \sqrt{2}\,t\,\hi + e^{5t}\,\hj - e^{-5t}\,\hk
\end{equation*}
Select the correct answer:
\begin{enumerate}[(a)]
\item\ \ 
$|\vv(t)| = \big(e^{5t} + e^{-5t}\big)$
\item\ \ 
$|\vv(t)| = \sqrt{10 + 5e^{t} + 5e^{-t}}$
\item\ \ 
$|\vv(t)| = \sqrt{10 + e^{10t} + e^{-10t}}$
\item\ \ 
$|\vv(t)| = 5\big(e^{5t} + e^{-5t}\big)$
\item\ \ 
$|\vv(t)| = 5\big(e^t + e^{-t}\big)$
\end{enumerate}
\end{question}

\begin{hint} 
Just compute $|\vv(t)|$. Note that $\big(e^{at}+e^{-at}\big)^2 
=e^{2at} + 2 + e^{-2at}$.
\end{hint}

\begin{answer} 
(d)
\end{answer}

\begin{solution}
We have
\begin{equation*}
\vv(t) = \vr'(t) 
      = 5 \sqrt{2}\,\hi + 5e^{5t}\,\hj +5 e^{-5t}\,\hk
\end{equation*}
and hence
\begin{equation*}
|\vv(t)| = |\vr'(t)| 
         = 5 \big|\sqrt{2}\,\hi + e^{5t}\,\hj + e^{-5t}\,\hk\big|
         = 5\sqrt{2+ e^{10t}+ e^{-10t}}
\end{equation*}
Since $2+ e^{10t}+ e^{-10t} = \big(e^{5t}+e^{-5t}\big)^2$, that's (d).
\end{solution}

%%%%%%%%%%%%%%%%%%%%%%%%%%%%%%%
\begin{question}
Find the velocity, speed and acceleration at time $t$ of
the particle whose position is 
\begin{equation*}
\vr(t)= a \cos t\,\hi+a\sin t\,\hj+ct\,\hk
\end{equation*}
Describe the path of the particle.
\end{question}

\begin{hint} 
To figure out what the path looks like, first concentrate on the $x$- and
$y$-coordinates.
\end{hint}

\begin{answer} 
$\text{velocity}=-a \sin t\,\hi+a\cos t\,\hj+c\,\hk$\qquad
$\text{speed}= \sqrt{a^2+c^2}$\qquad
$\text{acceleration}=-a \cos t\,\hi-a\sin t\,\hj$

The path is a helix with radius $a$ and with each turn having height $2\pi c$.
\end{answer}

\begin{solution} 
We are told that
\begin{equation*}
\vr(t)= a \cos t\,\hi+a\sin t\,\hj+ct\,\hk
\end{equation*}
So, by definition,
\begin{align*}
\text{velocity}&= \vv(t)= \vr'(t)=-a \sin t\,\hi+a\cos t\,\hj+c\,\hk\\
\text{speed}&=\diff{s}{t}(t) = |\vr'(t)| = \sqrt{a^2+c^2} \\
\text{acceleration}&=\va(t)= \vr''(t)=-a \cos t\,\hi-a\sin t\,\hj
\end{align*}
As $t$ runs over an interval of length $2\pi$, $(x,y)$ traces
out a circle of radius $a$ and $z$ increases by $2\pi c$.
The path is a helix with radius $a$ and with each turn having height $2\pi c$.
\end{solution}


%%%%%%%%%%%%%%%%%%%
\begin{question}[M317 2013D] %2

\begin{enumerate}[(a)]
\item
Let
\begin{equation*}
\vr(t) = \left(t^2 , 3, \tfrac{1}{3} t^3 \right) 
\end{equation*}
Find the unit tangent vector to this parametrized curve at $t = 1$, 
pointing in the direction of increasing $t$.
\item
Find the arc length of the curve from (a) between the points $(0, 3, 0)$ 
and $(1, 3, -\frac{1}{3})$.

\end{enumerate}
\end{question}

\begin{hint} 
Review \S\eref{CLP317}{sec:CurveCompendium} of the CLP-4 text.
The arc length should be positive.
\end{hint}

\begin{answer} 
(a) $\hat\vT(1) = \frac{(2,0,1)}{\sqrt{5}}$\qquad
(b) $\frac{1}{3}\big[5^{3/2}-8\big]$
\end{answer}

\begin{solution} (a)
Since $\vr'(t) = (2t,0,t^2)$, the specified unit tangent at $t=1$ is
\begin{equation*}
\hat\vT(1) = \frac{(2,0,1)}{\sqrt{5}}
\end{equation*}

\noindent (b)
We are to find the arc length between $\vr(0)$ and $\vr(-1)$. 
As $\diff{s}{t}=\sqrt{4t^2+t^4}$, the 
\begin{align*}
\text{arc length} &= \int_{-1}^0 \sqrt{4t^2+t^4}\ \dee{t}
\end{align*}
The integrand is even, so
\begin{align*}
\text{arc length} &= \int_0^1 \sqrt{4t^2+t^4}\ \dee{t}
=\int_0^1 t\sqrt{4+t^2}\ \dee{t}
=\Big[\tfrac{1}{3}{(4+t^2)}^{3/2}\Big]_0^1
=\tfrac{1}{3}\big[5^{3/2}-8\big]
\end{align*}

\end{solution}



%%%%%%%%%%%%%%%%%%%
\begin{question}\label{prob_s1.1:lemma113}
Using Lemma~\eref{CLP317}{lem:CVtanArclen}  in the CLP-4 text, 
find the arclength of $\vr(t)=\left(t,\sqrt{\frac{3}{2}}t^2,t^3\right)$ 
from $t=0$ to $t=1$.
\end{question}
\begin{hint}
From Lemma~\eref{CLP317}{lem:CVtanArclen} in the CLP-4 text, we know 
the arclength from $t=0$ to $t=1$ will be
\[\int_{0}^1\left| \diff{\vr}{t}(t)\right|\dee t\]
The notation looks a little confusing at first, but we can break it down piece by piece: $\diff{\vr}{t}(t)$ is a vector, whose components are functions of $t$. If we take its magnitude, we'll get one big function of $t$. That function is what we integrate. Before integrating it, however, we should simplify as much as possible.
\end{hint}
\begin{answer}
2
\end{answer}
\begin{solution}
By Lemma~\eref{CLP317}{lem:CVtanArclen}  in the CLP-4 text,  
the arclength of $\vr(t)$ from $t=0$ to $t=1$ is
$\int_{0}^1\left| \diff{\vr}{t}(t)\right|\dee t$. We'll calculate this in a few pieces to make the steps clearer.
\begin{align*}
\vr(t)&=\left(t,\sqrt{\frac{3}{2}}t^2,t^3\right)\\
\diff{\vr}{t}(t)&=\left(1,\sqrt{6}t,3t^2\right)\\
\left|\diff{\vr}{t}(t)\right|&=\sqrt{1^2+(\sqrt{6}t)^2+(3t^2)^2}=\sqrt{1+6t^2+9t^4}=\sqrt{(3t^2+1)^2}=3t^2+1\\
\int_{0}^1\left| \diff{\vr}{t}(t)\right|\dee t&=\int_0^1\left(3t^2+1 \right)\dee t=2
\end{align*}
\end{solution}
%%%%%%%%%%%%%%%%%%%


%%%%%%%%%%%%%%%%%%%%%%%%%%%%%%%
\begin{question}
Find the length of the parametric curve
\begin{equation*}
x=a\cos t\sin t\qquad
y=a\sin^2 t\qquad
z=bt
\end{equation*}
between $t=0$ and $t=T>0$.
\end{question}

%\begin{hint} 
%
%\end{hint}

\begin{answer} 
$\text{length}=\sqrt{a^2+b^2}\,T$
\end{answer}

\begin{solution} 
Since
\begin{align*}
x'(t)&=a\big[\cos^2 t-\sin^2 t\big]=a\cos 2t\\
y'(t)&=2a\sin t\cos t=a\sin 2t\\
z'(t)&=b
\end{align*}
we have
\begin{equation*}
\diff{s}{t}(t)
=\sqrt{x'(t)^2+y'(t)^2+z'(t)^2}=\sqrt{a^2+b^2}
\end{equation*}
As the speed $\diff{s}{t}(t)$ is constant, the length is just
$\diff{s}{t}\,T=\sqrt{a^2+b^2}\,T$.

\end{solution}


%%%%%%%%%%%%%%%%%%%%%%%%%%%%%%%%%%%%%%
\begin{question}
A particle's position at time $t$ is given by $\vr(t)=(t+\sin t, \cos t)$\footnote{The particle traces out a cycloid --- see Question~\ref{prob_s1.1_cycloid}}. What is the magnitude of the acceleration of the particle at time $t$?
\end{question}
\begin{hint}
$\vr(t)$ is the position of the particle, so its acceleration is $r''(t)$.
\end{hint}
\begin{answer}
1
\end{answer}
\begin{solution}
Since $\vr(t)$ is the position of the particle,  its acceleration is $r''(t)$.
\begin{align*}
\vr(t)&=(t+\sin t, \cos t)\\
\vr'(t)&=(1+\cos t,-\sin t)\\
\vr''(t)&=(-\sin t,-\cos t)\\
|\vr''(t)|&=\sqrt{\sin^2t+\cos^2t}=1
\end{align*}
The magnitude of acceleration is constant, but its \emph{direction} is changing, since $\vr''(t)$ is a vector with changing direction.
\end{solution}



%%%%%%%%%%%%%%%%%%%%%%%%%%%
\begin{question}[M317 2011D] %2
A curve in $\bbbr^3$ is given by the vector equation 
$\vr(t) = \left(2t \cos t, 2t \sin t,\frac{t^3}{3}\right)$

\begin{enumerate}[(a)]
\item
Find the length of the curve between $t = 0$ and $t = 2$.

\item
Find the parametric equations of the tangent line to the curve at $t = \pi$.

\end{enumerate}
\end{question}

\begin{hint} 
Review \S\eref{CLP317}{sec:CurveCompendium} of the CLP-4 text.
\end{hint}

\begin{answer} 
(a) $\nicefrac{20}{3}$\qquad
(b) $x(t) = -2\pi -2t,\ 
     y(t) =  -2\pi t,\ 
     z(t) = \nicefrac{\pi^3}{3} + \pi^2 t$ 
\end{answer}

\begin{solution} (a) The speed is
\begin{align*}
\diff{s}{t}(t)
=\big|\vr'(t)\big| & = \left|\left(2\cos t - 2t\sin t\,,\,
                                  2\sin t + 2t \cos t\,,\,
                                  t^2\right)\right| \\
                  &=\sqrt{\big(2\cos t - 2t\sin t\big)^2
                         +\big(2\sin t + 2t \cos t\big)^2
                         +t^4} \\
                 &= \sqrt{4+ 4t^2 +t^4} \\
                 &= 2+t^2
\end{align*}
so the length of the curve is
\begin{align*}
\text{length }
&=\int_0^2 \diff{s}{t}\,\dee{t}
  =\int_0^2 (2+t^2)\,\dee{t}
  = \left[2t +\frac{t^3}{3}\right]_0^2
  =\frac{20}{3}
\end{align*}

\noindent (b)
A tangent vector to the curve at 
$\vr(\pi)=\big(-2\pi\,,\, 0\,,\, \nicefrac{\pi^3}{3}\big)$ is
\begin{equation*}
\vr'(\pi) = \left(2\cos\pi - 2\pi\sin\pi\,,\,
                                  2\sin\pi + 2\pi \cos\pi\,,\,
                                  \pi^2\right)
 = (-2\,,\,-2\pi\,,\,\pi^2)
\end{equation*}
So parametric equations for the tangent line at $\vr(\pi)$ are
\begin{align*}
x(t) &= -2\pi -2t \\
y(t) &=  -2\pi t \\
z(t) &= \nicefrac{\pi^3}{3} + \pi^2 t 
\end{align*}
\end{solution}
%%%%%%%%%%%%%%%%%%%%%%%%%%%%%%%

\begin{question}[M317 2010D]  %1
Let $\vr(t) = \big(3 \cos t, 3 \sin t, 4t\big)$ be the position vector 
of a particle as a function of time $t \ge 0$.
\begin{enumerate}[(a)]
\item
Find the velocity of the particle as a function of time $t$.
\item
Find the arclength of its path between $t = 1$ and $t = 2$.
\end{enumerate}
\end{question}

\begin{hint} 
Review \S\eref{CLP317}{sec curve derivs} of the CLP-4 text.
\end{hint}

\begin{answer} 
(a) $\vr'(t) = \big(-3 \sin t, 3 \cos t, 4\big)$\qquad
(b) 5
\end{answer}

\begin{solution} (a) As $\vr(t) = \big(3 \cos t, 3 \sin t, 4t\big)$,
the velocity of the particle is
\begin{equation*}
\vr'(t) = \big(-3 \sin t, 3 \cos t, 4\big)
\end{equation*}

\noindent (b) As $\diff{s}{t}$, the rate of change of arc length per unit time,
is
\begin{equation*}
\diff{s}{t}(t) = |\vr'(t)| = \big|\big(-3 \sin t, 3 \cos t, 4\big)\big|
  =5
\end{equation*}
the arclength of its path between $t = 1$ and $t = 2$ is
\begin{equation*}
\int_1^2\dee{t}\ \diff{s}{t}(t) 
=\int_1^2\dee{t}\ 5
=5
\end{equation*}
\end{solution}

%%%%%%%%%%%%%%%%%%%%%%%%%%%%%%%
\begin{question}
The plane $\ z=2x+3y\ $ intersects the cylinder 
$\ x^2+y^2=9\ $ in an ellipse. 
\begin{enumerate}[(a)]
\item
Find a parametrization of the ellipse. 
\item
Express the circumference of this ellipse as an integral. 
You need not evaluate the integral\footnote{The indefinite integral involved is one of a class of integrals called elliptic integrals because of their connections to arc lengths of ellipses. In general, elliptic integrals cannot be expressed in terms of elementary functions. You can easily find discussions of elliptic integrals using your favourite search engine.}.
\end{enumerate}
\end{question}

\begin{hint} 
(a) First parametrize $x^2+y^2=9$.
\end{hint}

\begin{answer} 
(a) $x(\theta)=3\cos\theta,\ 
y(\theta)=3\sin\theta,\ 
z(\theta)=6\cos\theta+9\sin\theta,\ 
0\le\theta\le 2\pi$

(b) $s=\int_0^{2\pi} \sqrt{45+45\cos^2\theta-108\sin\theta\cos\theta}\,d\theta$

\end{answer}

\begin{solution} (a)
We can parametrize the circle  $\ x^2+y^2=9\ $ as 
$x(\theta)=3\cos\theta$, $y(\theta)=3\sin\theta$ with 
$\theta$ running from $0$ to $2\pi$. As $z=2x+3y$, the 
ellipse can be parametrized by
\begin{equation*}
x(\theta)=3\cos\theta,\ 
y(\theta)=3\sin\theta,\ 
z(\theta)=2x(\theta)+3 y(\theta)
        =6\cos\theta+9\sin\theta,\ 
0\le\theta\le 2\pi
\end{equation*}

(b)
As
\begin{align*}
\diff{s}{\theta}
 &=\sqrt{x'(\theta)^2+y'(\theta)^2+z'(\theta)^2}\\
 &=\sqrt{9\sin^2\theta+9\cos^2\theta+36\sin^2\theta
        +81\cos^2\theta-108\sin\theta\cos\theta}\\
&=\sqrt{45+45\cos^2\theta-108\sin\theta\cos\theta}
\end{align*}
the circumference is
\begin{equation*}
s=\int_0^{2\pi} \sqrt{45+45\cos^2\theta-108\sin\theta\cos\theta}\,\dee{\theta}
\end{equation*}
\end{solution}



%%%%%%%%%%%%%%%%%%%%%%%%%%
\begin{question}[M317 2007A] %2
	Consider the curve
	\begin{equation*}
	\vr(t) = \frac{1}{3}\cos^3 t\,\hi +\frac{1}{3} \sin^3 t\,\hj + \sin^3 t\,\hk
	\end{equation*}
	\begin{enumerate}[(a)]
		\item
		Compute the arc length of the curve from $t = 0$ to $t = \frac{\pi}{2}$.
		\item
		Compute the arc length of the curve from $t = 0$ to $t = \pi$.
	\end{enumerate}
\end{question}

\begin{hint} 
	If you got the answer $0$ in part (b), you dropped some absolute value signs.
\end{hint}

\begin{answer} 
	(a) $\frac{1}{27}\big(10\sqrt{10}-1\big)$\qquad
	(b) $\frac{2}{27}\big(10\sqrt{10}-1\big)$
\end{answer}

\begin{solution} (a)
	As
	\begin{align*}
	\vr'(t) & = -\sin t\cos^2 t\,\hi + \sin^2 t\cos t\,\hi + 3\sin^2 t\cos t\,\hk
	= \sin t\cos t\big(-\cos t\,\hi +\sin t\,\hj +3\sin t\,\hk\big) \\
	\diff{s}{t}(t) & = |\sin t\cos t|\sqrt{\cos^2 t + \sin ^2 t + 9\sin^2 t}
	= |\sin t\cos t|\sqrt{1+ 9\sin^2 t}
	\end{align*}
	the arclength from $t = 0$ to $t = \frac{\pi}{2}$ is
	\begin{align*}
	\int_0^{\pi/2} \diff{s}{t}(t)\,\dee{t}
	&=\int_0^{\pi/2} \sin t\cos t \sqrt{1+ 9\sin^2 t}\,\dee{t} \\
	&=\frac{1}{18}\int_1^{10} \sqrt{u}\ \dee{u} \qquad
	\text{with } u = 1+ 9\sin^2 t,\ \dee{u}  = 18\sin t\cos t\,\dee{t}\\
	&=\frac{1}{18}\Big[\frac{2}{3}u^{3/2}\Big]_1^{10}\\
	&=\frac{1}{27}\big(10\sqrt{10}-1\big)
	\end{align*}
	
	(b) The arclength from $t = 0$ to $t = \pi$ is
	\begin{align*}
	\int_0^{\pi} \diff{s}{t}(t)\,\dee{t}
	&=\int_0^{\pi} |\sin t\cos t| \sqrt{1+ 9\sin^2 t}\,\dee{t} 
	\qquad\text{Don't forget the absolute value signs!}\\
	&=2\int_0^{\pi/2} |\sin t\cos t| \sqrt{1+ 9\sin^2 t}\,\dee{t} 
	= 2\int_0^{\pi/2} \sin t\cos t \sqrt{1+ 9\sin^2 t}\,\dee{t} 
	\end{align*}
	since the integrand is invariant under $t\rightarrow\pi-t$. So the arc length
	from $t = 0$ to $t = \pi$ is just twice the arc length from part (a), namely $\frac{2}{27}\big(10\sqrt{10}-1\big)$.
	
\end{solution}
%%%%%%%%%%%%%%%%%%%%%%%%%%%%%%%
\begin{question}[M317 2017D] %2
Let $\vr(t)=\big(\frac{1}{3}t^3,\frac{1}{2}t^2,\frac{1}{2}t\big)$,
$t\ge 0$. Compute $s(t$), the arclength of the curve at time
$t$.
\end{question}

%\begin{hint} 
%\end{hint}

\begin{answer} 
$s(t)=\frac{t^3}{3} +\frac{t}{2}$
\end{answer}

\begin{solution} 
Since
\begin{align*}
\vr(t)&= \frac{t^3}{3}\,\hi + \frac{t^2}{2}\,\hj +  \frac{t}{2}\,\hk\\
\vr'(t)&= t^2\,\hi + t\,\hj + \frac{1}{2}\,\hk \\
\diff{s}{t}(t)=|\vr'(t)|&=\sqrt{t^4+t^2+\frac{1}{4}}
=\sqrt{\Big(t^2+\frac{1}{2}\Big)^2}=t^2+\frac{1}{2}
\end{align*}
the length of the curve is
\begin{align*}
s(t)=\int_0^t \diff{s}{t}(u)\,\dee{u}
=\int_0^t \Big(u^2+\frac{1}{2}\Big)\,\dee{u}
=\frac{t^3}{3} +\frac{t}{2}
\end{align*}
\end{solution}


%%%%%%%%%%%%%%%%%%%%%%%%%%%%%%%
\begin{question}[M317 2011A] %3
Find the arc length of the curve 
$\vr(t) = \big(t^m\,,\, t^m\,,\, t^{3m/2}\big)$ for $0 \le a \le t \le b$, 
and where $m > 0$.
Express your result in terms of $m$, $a$, and $b$. 
\end{question}

\begin{hint} 
The integral you get can be evaluated with
a simple substitution. You may want to factor the integrand first.
\end{hint}

\begin{answer} 
$\frac{8}{27}\Big[\Big(2 + \frac{9}{4}b^m\Big)^{3/2}
                   -\Big(2 + \frac{9}{4}a^m\Big)^{3/2}\Big]$
\end{answer}

\begin{solution} 
Since
\begin{align*}
\vr(t) & = t^m\,\hi + t^m\,\hj + t^{3m/2}\,\hk \\
\vr'(t) &= mt^{m-1}\,\hi + mt^{m-1}\,\hj +\frac{3m}{2}t^{3m/2-1}\,\hk \\
\diff{s}{t} = |\vr'(t)| & = \sqrt{ 2m^2 t^{2m-2} +\frac{9m^2}{4} t^{3m-2} } 
= mt^{m-1}\sqrt{2 + \frac{9}{4}t^m }
\end{align*}
the arc length is
\begin{align*}
\int_a^b \diff{s}{t}(t)\,\dee{t}
&=\int_a^b mt^{m-1}\sqrt{2 + \frac{9}{4}t^m }\ \dee{t} \\
&=\frac{4}{9}\int_{2 + \frac{9}{4}a^m}^{2 + \frac{9}{4}b^m}\sqrt{u}\,\dee{u}
\qquad\text{with } u = 2 + \frac{9}{4}t^m,\ \dee{u} = \frac{9m}{4}t^{m-1}\\
&=\frac{4}{9}\Big[\frac{2}{3}u^{3/2}\Big]_{2 + \frac{9}{4}a^m}
                                         ^{2 + \frac{9}{4}b^m} \\
&=\frac{8}{27}\Big[\Big(2 + \frac{9}{4}b^m\Big)^{3/2}
                   -\Big(2 + \frac{9}{4}a^m\Big)^{3/2}\Big]
\end{align*} 
\end{solution}


%%%%%%%%%%%%%%%%%%%%%%%%%%%%%%%
\begin{question}
 Let $C$ be the part of the curve of intersection of the 
parabolic cylinder $x = y^2$ and the hyperbolic  paraboloid $3z = 2xy$
with $y\ge 0$.
\begin{enumerate}[(a)] 
\item
   Write a vector parametric equation for $C$ using $x$ as the parameter. 
\item 
   Find the length of the part of $C$ between the origin and 
   the point $(9, 3, 18)$. 
\item
   A particle moves along $C$ in the direction for which $x$ is 
   increasing.  If the particle  moves with constant speed 9, find 
   its velocity vector when it is at  the point $(1, 1, \frac{2}{3})$. 
\item
   Find the acceleration vector of the particle of part (c) 
   when it is at  the point $(1, 1, \frac{2}{3})$. 
\end{enumerate}
\end{question}

\begin{hint} 
(b) $\frac{1}{4x}+1+x$ is a perfect square.

(c), (d) Let 
\begin{itemize}\itemsep1pt \parskip0pt \parsep0pt %\itemindent-15pt
\item
  $\vr(x)$ be the position of the particle when its first coordinate is $x$,
\item
 $\vR(t)$ be the position of the particle at time $t$, and 
\item 
  $x(t)$ be the $x$--coordinate of the particle at time $t$.
\end{itemize}
Then $\vR(t) = \vr\big(x(t)\big)$. We are told $|\vR'(t)|=9$ for all $t$.
\end{hint}

\begin{answer} 
(a) $\vr(x)=x\,\hi+\sqrt{x}\,\hj+\frac{2}{3}x^{3/2}\,\hk$\quad
(b) $21$\quad
(c) $6\,\hi+3\,\hj+6\,\hk$\quad
(d) $-6\,\hi-12\,\hj+12\,\hk$
\end{answer}

\begin{solution} 
(a) 
Since $y=\sqrt{x}$ and $z=\frac{2}{3}xy=\frac{2}{3}x^{3/2}$,
\begin{align*}
\vr(x)&=x\,\hi+\sqrt{x}\,\hj+\frac{2}{3}x^{3/2}\,\hk
\end{align*}
For the remaining parts of this problem we will also need
\begin{align*}
\vr'(x)&=\hi+\frac{1}{2\sqrt{x}}\,\hj+\sqrt{x}\,\hk\\
\vr''(x)&=-\frac{1}{4x^{3/2}}\,\hj+\frac{1}{2\sqrt{x}}\,\hk\\
\diff{s}{x}=|\vr'(x)|
&=\sqrt{1+\frac{1}{4x}+x}
=\sqrt{\left(\frac{1}{2\sqrt{x}}+\sqrt{x}\right)^2}
=\frac{1}{2\sqrt{x}}+\sqrt{x}\\
\diff{s}{x}(1)&=\frac{3}{2}
\end{align*}

(b)
\begin{equation*}
\int_C \ ds
=\int_0^9 \diff{s}{x}\ \dee{x}
=\int_0^9 \left(\frac{1}{2\sqrt{x}}+\sqrt{x}\right)\ \dee{x}
=\left[\sqrt{x}+\frac{2}{3}x^{3/2}\right]_0^9
=3+18
=21
\end{equation*}

(c) 
Denote by
\begin{itemize}\itemsep1pt \parskip0pt \parsep0pt %\itemindent-15pt
\item
  $\vr(x)$ the position of the particle when its first coordinate is $x$,
\item
 $\vR(t)$ the position of the particle at time $t$, 
\item 
  $x(t)$ the $x$--coordinate of the particle at time $t$, and
\item 
  $s(x)$ the arc length of the curve from the origin to $\vr(x)$.
\end{itemize}
We are told that $|\vR'(t)|=9$ for all $t$.
So
\begin{align*}
\vR(t)=\vr\big(x(t)\big)&\implies
\vR'(t)=\vr'\big(x(t)\big)\diff{x}{t}(t) \\
&\implies
9=|\vR'(t)|=\diff{s}{x}\big(x(t)\big) \diff{x}{t}(t)
=\left(\frac{1}{2\sqrt{x(t)}}+\sqrt{x(t)}\right)\diff{x}{t}(t)
\end{align*}
In particular, if the particle is at $(1,1,\frac{2}{3})$ at time $0$, then $x(0)=1$
and
\begin{equation*}
9=\left(\frac{1}{2\sqrt{1}}+\sqrt{1}\right)\diff{x}{t}(0)
\implies \diff{x}{t}(0)=6
\end{equation*}
so that
\begin{equation*}
\vR'(0)=\vr'(1)\diff{x}{t}(0)
=\left(\hi+\frac{1}{2}\,\hj+\hk\right)6
=\ 6\,\hi+3\,\hj+6\,\hk
\end{equation*}

(d)
By the product and chain rules,
\begin{equation*}
\vR'(t)=\vr'\big(x(t)\big)\diff{x}{t}(t)\implies
\vR''(t)=\vr''\big(x(t)\big)\left(\diff{x}{t}(t)\right)^2
+\vr'\big(x(t)\big)\difftwo{x}{t}(t)
\end{equation*}
We saw in part (c) that 
$9=|\vR'(t)|=\Big(\frac{1}{2\sqrt{x(t)}}+\sqrt{x(t)}\Big)\diff{x}{t}(t)$
so that 
\begin{equation*}
\diff{x}{t}(t)=9\left(\frac{1}{2\sqrt{x(t)}}+\sqrt{x(t)}\right)^{-1}
\end{equation*}
Differentiating that gives
\begin{equation*}
\frac{d^2x}{dt^2}(t)=-9\left(\frac{1}{2\sqrt{x(t)}}+\sqrt{x(t)}\right)^{-2}
\left(-\frac{1}{4x(t)^{3/2}}+\frac{1}{2\sqrt{x(t)}}\right)\diff{x}{t}(t)
\end{equation*}
In particular, when $t=0$, $x(0)=1$ and $\diff{x}{t}(0)=6$
\begin{equation*}
\difftwo{x}{t}(0)=-9\left(\frac{3}{2}\right)^{-2}
\left(\frac{1}{4}\right)6=-6
\end{equation*}
so
\begin{align*}
\vR''(0)&=\vr''(1)\ \big(6\big)^2+\vr'(1)\big(-6\big)
=36\left(-\frac{1}{4}\,\hj+\frac{1}{2}\,\hk\right)
-6\left(\hi+\frac{1}{2}\,\hj+\hk\right) \\
&=-6\,\hi-12\,\hj+12\,\hk
\end{align*}
\end{solution}




%%%%%%%%%%%%%%%%%%%
\begin{question}
If a particle has constant mass $m$, position $\vr$, and is moving with velocity $\vv$, then its angular momentum is $\vL=m(\vr\times\vv)$. 

For a particle with mass $m=1$ and position function $\vr=(\sin t, \cos t, t)$, find $\left|\diff{\vL}{t} \right|$.

\end{question}
\begin{hint} 
Given the position of a particle, you can find its velocity.
\end{hint}
\begin{answer}
$|t|$
\end{answer}
\begin{solution}
Given the position of the particle, we can find its velocity:
\[\vv(t)=\vr'(t)=(\cos t, -\sin t , 1)\]
Applying the given formula,
\[\vL(t)=\vr \times \vv=(\sin t , \cos t , t) \times (\cos t, -\sin t , 1).\]
\begin{description}
\item[\textbf{Solution 1:}]
We can first compute the cross product, then differentiate:
\begin{align*}
\vL(t)& = (\cos t + t\sin t)\hi + (t\cos t - \sin t)\hj-\hk\\
\vL'(t)&=t\cos t\,\hi -t\sin t\, \hj\\
|\vL'(t)|&=\sqrt{t^2(\sin^2 t + \cos^2 t)}=\sqrt{t^2}=|t|
\end{align*}
\item[\textbf{Solution 2:}]
Using the product rule:
\begin{align*}
\vL'(t)&=\vr'(t)\times \vv(t) + \vr(t) \times \vv'(t)\\
&=\underbrace{\vr'(t)\times \vr'(t)}_{0} + \vr(t) \times \vv'(t)\\
&= (\sin t ,\cos t,t)\times(-\sin t , -\cos t, 0)\\
&=t\cos t\,\hi-t\sin t\,\hj\\
|\vL'(t)|&=\sqrt{t^2\cos^2 t + t^2\sin t^2  }=|t|
\end{align*}
\end{description}
\end{solution}



%%%%%%%%%%%%%%%%%
\subsection*{\Application}
%%%%%%%%%%%%%%%%%


%%%%%%%%%%%%%%%%%%%%%%%%%%%%
\begin{question}[M317 2000A] %1
 A particle moves along the curve $\cC$ of intersection of
the surfaces $z^2=12y$ and $18x=yz$ in the upward direction. When the particle
is at $(1,3,6)$ its velocity $\vv$ and acceleration $\va$ are given by
$$
\vv =6\,\hi+12\,\hj+12\,\hk\qquad
\va = 27\,\hi+30\,\hj+6\,\hk
$$
\begin{enumerate}[(a)]
\item
 Write a vector parametric equation for $\cC$ using 
$u=\frac{z}{6}$ as a parameter.
\item
 Find the length of $\cC$ from $(0,0,0)$ to $(1,3,6)$.
\item
 If $u=u(t)$ is the parameter value for the particle's position
at time $t$, find $\diff{u}{t}$ when the particle is at $(1,3,6)$.
\item
 Find $\difftwo{u}{t}$ when the particle is at $(1,3,6)$.
\end{enumerate}
\end{question}

\begin{hint} 
If $\vr(u)$ is the parametrization of $\cC$ by $u$, then
the position of the particle at time $t$ is $\vR(t) = \vr\big(u(t)\big)$.
\end{hint}

\begin{answer} 
(a)  $\vr(u)=u^3\,\hi+3u^2\,\hj+6u\,\hk$\qquad
(b)  $7$\qquad
(c)  $2$\qquad
(d)  $1$
\end{answer}

\begin{solution}
(a) 
Since $z=6u$, $y=\frac{z^2}{12}=3u^2$ and $x=\frac{yz}{18}=u^3$,
\begin{align*}
\vr(u)&=u^3\,\hi+3u^2\,\hj+6u\,\hk
\end{align*}

(b)
\begin{align*}
\vr'(u)&=3u^2\,\hi+6u\,\hj+6\,\hk\\
\vr''(u)&=6u\,\hi+6\,\hj\\
\diff{s}{u}(u)=|\vr'(u)|
&=\sqrt{9u^4+36u^2+36}=3\big(u^2+2\big)\\
%\diff{s}{u}(1)&=9
\end{align*}


$$
\int_\cC \ ds
=\int_0^1 \diff{s}{u}\ \dee{u}
=\int_0^1 3\big(u^2+2\big)\ \dee{u}
=\big[u^3+6u\big]_0^1
=7
$$

(c) Denote by $\vR(t)$ the position of the particle at time $t$.
Then
$$
\vR(t)=\vr\big(u(t)\big)\implies
\vR'(t)=\vr'\big(u(t)\big)\diff{u}{t}
$$
In particular, if the particle is at $(1,3,6)$ at time $t_1$, then $u(t_1)=1$
and
$$
6\,\hi+12\,\hj+12\,\hk=\vR'(t_1)=\vr'(1)\diff{u}{t}(t_1)
=\big(3\,\hi+6\,\hj+6\,\hk\big)\diff{u}{t}(t_1)
$$
which implies that $\diff{u}{t}(t_1)=2$.

(d) By the product and chain rules,
$$
\vR'(t)=\vr'\big(u(t)\big)\diff{u}{t}\implies
\vR''(t)=\vr''\big(u(t)\big)\Big(\diff{u}{t}\Big)^2
+\vr'\big(u(t)\big)\difftwo{u}{t}
$$
In particular,
\begin{align*}
27\,\hi+30\,\hj+6\,\hk
&=\vR''(t_1)=\vr''(1)\Big(\diff{u}{t}(t_1)\Big)^2
+\vr'\big(1\big)\difftwo{u}{t}(t_1) \\
&=\big(6\,\hi+6\,\hj\big)2^2+\big(3\,\hi+6\,\hj+6\,\hk\big)\difftwo{u}{t}(t_1)
\end{align*}
Simplifying
$$
3\,\hi+6\,\hj+6\,\hk
=\big(3\,\hi+6\,\hj+6\,\hk\big)\difftwo{u}{t}(t_1)
\implies \difftwo{u}{t}(t_1)=1
$$
\end{solution}


%%%%%%%%%%%%%%%%%%%%%%%%%%%
\begin{question}[M317 2008A] %2
A particle of mass $m = 1$ has position $\vr_0 = \frac{1}{2}\,\hk$ 
and velocity $\vv_0 =\frac{\pi^2}{2}\,\hi$ at time $0$.
It moves under a force
\begin{equation*}
\vF(t) = -3t\,\hi + \sin t\,\hj + 2e^{2t}\,\hk.
\end{equation*}
\begin{enumerate}[(a)]
\item
Determine the position $\vr(t)$ of the particle depending on $t$.
\item
At what time after time $t = 0$ does the particle cross the plane 
$x = 0$ for the first time?
\item
What is the velocity of the particle when it crosses the plane $x = 0$ 
in part (b)?
\end{enumerate}
\end{question}

\begin{hint} 
By Newton's law, $\vF=m\va$.
\end{hint}

\begin{answer} 
(a) $\vr(t) = \big(\frac{\pi^2 t}{2}-\frac{t^3}{2}\big)\,\hi + 
              (t- \sin t)\,\hj + \left(\frac{1}{2}e^{2t}-t\right)\,\hk$ \qquad
(b) $t=\pi$ 

(c) $-\pi^2\,\hi +2\,\hj + \big(e^{2\pi}-1\big)\,\hk$
\end{answer}

\begin{solution} (a)
According to Newton,
\begin{equation*}
m\vr''(t) = \vF(t)\qquad\text{so that}\qquad
\vr''(t) = -3t\,\hi + \sin t\,\hj + 2e^{2t}\,\hk
\end{equation*}
Integrating once gives
\begin{align*}
\vr'(t) = -3\frac{t^2}{2}\,\hi - \cos t\,\hj + e^{2t}\,\hk +\vc
\end{align*}
for some constant vector $\vc$. We are told that $\vr'(0)=\vv_0
=\frac{\pi^2}{2}\,\hi$. This forces $\vc=\frac{\pi^2}{2}\,\hi+\hj-\hk$
so that
\begin{align*}
\vr'(t) = \left(\frac{\pi^2}{2}-\frac{3t^2}{2}\right)\,\hi +(1- \cos t)\,\hj + \big(e^{2t}-1\big)\,\hk 
\end{align*}
Integrating a second time gives
\begin{align*}
\vr(t) = \left(\frac{\pi^2 t}{2}-\frac{t^3}{2}\right)\,\hi +(t- \sin t)\,\hj 
 + \left(\frac{1}{2}e^{2t}-t\right)\,\hk  + \vc
\end{align*}
for some (other) constant vector $\vc$.  We are told that $\vr(0)=\vr_0
=\frac{1}{2}\,\hk$. This forces $\vc=\vZero$
so that
\begin{align*}
\vr(t) = \left(\frac{\pi^2 t}{2}-\frac{t^3}{2}\right)\,\hi +(t- \sin t)\,\hj 
 + \left(\frac{1}{2}e^{2t}-t\right)\,\hk 
\end{align*}

(b) The particle is in the plane $x=0$ when
\begin{align*}
0=\left(\frac{\pi^2 t}{2}-\frac{t^3}{2}\right)
 =\frac{t}{2}(\pi^2-t^2) 
\iff t=0, \pm\pi
\end{align*}
So the desired time is $t=\pi$.

(c) At time $t=\pi$, the velocity is
\begin{align*}
\vr'(\pi) 
&= \left(\frac{\pi^2}{2}-\frac{3\pi^2}{2}\right)\,\hi +(1- \cos\pi)\,\hj +                 \big(e^{2\pi}-1\big)\,\hk \\
&= -\pi^2\,\hi +2\,\hj + \big(e^{2\pi}-1\big)\,\hk
\end{align*}

\end{solution}
%%%%%%%%%%%%%%%%%%%%%%%%%%%%
\begin{question}[M317 1999A] %1
 Let $C$ be the curve of intersection of
the surfaces $y=x^2$ and $z=\frac{2}{3}x^3$.
A particle moves along $C$ with constant speed such that $\diff{x}{t}>0$.
The particle is at $(0,0,0)$ at time $t=0$ and is at $(3,9,18)$ at time
$t=\frac{7}{2}$.
\begin{enumerate}[(a)]
\item
Find the length of the part of $C$ between $(0,0,0)$ and $(3,9,18)$.
\item
Find the constant speed of the particle.
\item
Find the velocity of the particle when it is at $\big(1,1,\frac{2}{3}\big)$.
\item
Find the acceleration of the particle when it is at $\big(1,1,\frac{2}{3}\big)$.
\end{enumerate}
\end{question}

\begin{hint} 
Denote by $\vr(x)$ the parametrization of $C$ by $x$. If the $x$-coordinate
of the particle at time $t$ is $x(t)$, then the position of the particle at time $t$ is $\vR(t)=\vr\big(x(t)\big)$. Also, though the particle is moving at a constant speed, it doesn't necessarily have a constant value of $\diff{\vx}{t}$.
\end{hint}

\begin{answer} 
(a) $21$\qquad
(b) $6$\qquad
(c) $2\hi+4\,\hj+4\,\hk$\qquad
(d) $-\frac{8}{3}\big(2\hi+\,\hj-2\,\hk\big)$
\end{answer}

\begin{solution}
(a) Parametrize $C$ by $x$. Since $y=x^2$ and $z=\frac{2}{3}x^3$,
\begin{align*}
\vr(x)&=x\,\hi+x^2\,\hj+\frac{2}{3}x^3\,\hk\\
\vr'(x)&=\hi+2x\,\hj+2x^2\,\hk\\
\vr''(x)&=2\,\hj+4x\,\hk\\
\diff{s}{x}
&=|\vr'(x)|
=\sqrt{1+4x^2+4x^4}=1+2x^2
\end{align*}
and
$$
\int_C \ \dee{s}
=\int_0^3 \diff{s}{x}\ \dee{x}
=\int_0^3 \big(1+2x^2\big)\ \dee{x}
={\Big[x+\frac{2}{3}x^3\Big]}_0^3
=21
$$

(b) The particle travelled a distance of 21 units in $\frac{7}{2}$
time units. This corresponds to a speed of $\frac{21}{7/2}=6$.

(c) Denote by $\vR(t)$ the position of the particle at time $t$.
Then
$$
\vR(t)=\vr\big(x(t)\big)\implies
\vR'(t)=\vr'\big(x(t)\big)\diff{x}{t}
$$
By parts (a) and (b) and the chain rule
$$
6=\diff{s}{t}=\diff{s}{x}\diff{x}{t}=(1+2x^2)\diff{x}{t}
\implies \diff{x}{t}=\frac{6}{1+2x^2}
$$
In particular, the particle is at $\big(1,1,\frac{2}{3}\big)$ at $x=1$.
At this time $\diff{x}{t}=\frac{6}{1+2\times 1}=2$ and
$$
\vR'=\vr'\big(1\big)\diff{x}{t}
=\big(\hi+2\,\hj+2\,\hk\big)2
=2\hi+4\,\hj+4\,\hk
$$

(d) By the product and chain rules,
$$
\vR'(t)=\vr'\big(x(t)\big)\diff{x}{t}\implies
\vR''(t)=\vr''\big(x(t)\big){\Big(\diff{x}{t}\Big)}^2
+\vr'\big(x(t)\big)\difftwo{x}{t}
$$
Applying $\diff{\hfill}{t}$ to $6=\big(1+2x(t)^2\big)\diff{x}{t}(t)$ gives
$$
0=4x{\Big(\diff{x}{t}\Big)}^2+(1+2x^2)\difftwo{x}{t}
$$
In particular, when $x=1$ and $\diff{x}{t}=2$,
 $0=4\times 1\big(2\big)^2+(3)\difftwo{x}{t}$ gives
$\difftwo{x}{t}=-\frac{16}{3}$ and
$$
\vR''=\big(2\,\hj+4\,\hk\big)\big(2\big)^2
-\big(\hi+2\,\hj+2\,\hk\big)\frac{16}{3}
= -\frac{8}{3}\big(2\hi+\,\hj-2\,\hk\big)
$$

\end{solution}







%%%%%%%%%%%%%%%%%%%
\begin{question}
A camera mounted to a pole can swivel around in a full circle. It is tracking an object whose position at time $t$ seconds is $x(t)$ metres east of the pole, and $y(t)$ metres north of the pole.

In order to always be pointing directly at the object, how fast should the camera be programmed to rotate at time $t$? (Give your answer in terms of $x(t)$ and $y(t)$ and their derivatives, in the units rad/sec.)
\end{question}
\begin{hint} 
The question is already set up as an $xy$-plane, with the camera at the origin, so the vector in the direction the camera is pointing is $(x(t),y(t))$. Let $\theta$ be the angle the camera makes with the positive $x$-axis (due east). The tangent function gives a clean-looking relation between $\theta(t)$, $x(t)$, and $y(t)$.
\end{hint}
\begin{answer}
$\frac{x(t)y'(t)-y(t)x'(t)}{x^2+y^2}$
\end{answer}
\begin{solution}
The question is already set up as an $xy$-plane, with the camera at the origin, so the vector in the direction the camera is pointing is $(x(t),y(t))$. Let $\theta$ be the angle the camera makes with the positive $x$-axis (due east). The camera, the object, and the due-east direction (positive $x$-axis) make a right triangle.
\begin{center}
\begin{tikzpicture}
\YEaaxis{.5}{4}{.5}{3}
\draw[thick, ->] (0,0)--(3,2)node[xshift=1mm,yshift=.67mm,vertex, label=right:{object}]{};
\draw (1,0) arc (0:32:10mm) node[midway, right]{$\theta$};
\YExcoord{3}{x(t)}
\YEycoord{2}{y(t)}
\draw (0,0) node[vertex, label=below left:{camera}]{};
\end{tikzpicture}
\end{center}
\begin{align*}
\tan\theta&=\frac{y}{x}\intertext{Differentiating implicitly with respect to $t$:}
\sec^2\theta\,\diff{\theta}{t}&=\frac{xy'-yx'}{x^2}\\
\diff{\theta}{t}&=\cos^2\theta\left(\frac{xy'-yx'}{x^2}\right)=\left(\frac{x}{\sqrt{x^2+y^2}}\right)^2\left(\frac{xy'-yx'}{x^2}\right)=\frac{xy'-yx'}{{x^2+y^2}}
\end{align*}
\end{solution}

%%%%%%%%%%%%%%%%%%%
\begin{question}
A pipe of radius 3 follows the path of the curve $\vr(t)=(\frac{2\sqrt2}{3}t^{3/2}~,~\frac12t^2~,~t+2)$, for $0 \le t \le 10$. 

What is the volume inside the pipe? What is the surface area of the pipe?
\end{question}
\begin{hint}
Usng the Theorem of Pappus, the surface area and volume of this pipe are the same as that of a straight pipe with the same length and radius.
\end{hint}
\begin{answer}
Volume: $540 \pi$ \qquad Surface area: $360\pi$
\end{answer}
\begin{solution}
Using the Theorem of Pappus, we can calculate the surface area and volume of a pipe with the same length and radius as this pipe. So, we need to find the length of the pipe, $L$. 
\begin{align*}
\diff{\vr}{t}&=(\sqrt{2t},t,1)\\
\left| \diff{\vr}{t}\right|&=\sqrt{2t+t^2+1}=|t+1|\\
L&=\int_0^{10}(t+1)\dee{t}=60
\end{align*}

A pipe with radius 3 and length 60 has surface area $60(2\pi\cdot 3)=360\pi$ 
and volume $60(\pi \cdot3^2)=540\pi$.

\end{solution}
%%%%%%%%%%%%%%%%%%%

%%%%%%%%%%%%%%%%%%%
\begin{question}
A wire of total length 1000cm is formed into a flexible coil that is a 
circular helix. If there are 10 turns to each centimetre of height and the 
radius of the helix is 3 cm, how tall is the coil?
\end{question}

\begin{hint} 
A helix can be parametrized by 
$\vr(\theta)=a\cos\theta\,\hi+a\sin\theta\,\hj+b\theta\,\hk$.
\end{hint}

\begin{answer}
$\dfrac{50}{\pi\sqrt{9+\frac{1}{400\pi^2}}}\approx 5.3\textrm{ cm}$
\end{answer}

\begin{solution}
In general a helix can be parametrized by
\begin{equation*}
\vr(\theta)=a\cos\theta\,\hi+a\sin\theta\,\hj+b\theta\,\hk
\end{equation*}
Our first task is to determine $a$ and $b$.
The radius of the helix is $3$ cm, so $a=3$ cm. 
After 10 turns (i.e. $\theta=20\pi$) the height, $b\theta$,
is 1 cm. So $b(20\pi)=1$ and $b=\frac{1}{20\pi}$cm/rad. 
Thus 
$\vr(\theta)=3\cos\theta\,\hi+3\sin\theta\,\hj+\frac{1}{20\pi}\theta\hk$.

With each full turn of the helix (i.e. each increase of $\theta$ by $2\pi$)
the height of the helix increases by $2\pi b=\frac{1}{10}\textrm{cm}$.
So if we can determine the length of wire in one full turn of the helix, 
we can easily determine how many turns the helix goes through in total,
and from that we can determine the total height of the helix.

As $\vr'(\theta)=-3\sin\theta\,\hi+3\cos\theta\,\hj+\frac{1}{20\pi}\hk$
we have $\diff{s}{\theta}=\big|\vr'(\theta)\big|=\sqrt{9+\frac{1}{400\pi^2}}$.
So the length of one full turn of the helix is
\begin{align*}
\int_0^{2\pi}\sqrt{9+\frac{1}{400\pi^2}}\ \dee{\theta}
&=2\pi\sqrt{9+\frac{1}{400\pi^2}}
\end{align*}
and 1000cm of wire generates
\begin{equation*}
\frac{1000}{2\pi\sqrt{9+\frac{1}{400\pi^2}}}
=\frac{500}{\pi\sqrt{9+\frac{1}{400\pi^2}}}
\end{equation*}
turns. Each turn adds $\frac{1}{10}\textrm{cm}$ to the height, so the total height is
\begin{equation*}
\frac{500}{\pi\sqrt{9+\frac{1}{400\pi^2}}}\cdot \frac{1}{10}
=\frac{50}{\pi\sqrt{9+\frac{1}{400\pi^2}}}
\approx 5.3\textrm{ cm}
\end{equation*}

\emph{Remark.} 
We can check that this answer is reasonable by taking advantage of the fact
that each coil adds only a very small height (relative to the radius).
So we expect the length of one coil to be \emph{about} the same as the circumference of a circle of the same radius, namely $6\pi$. 
If we were making actual circles of the wire, there would be $\frac{1000}{6\pi}$ of them. Stacking up at 10 per centimetre, 
this would make a pile of height $\frac{1000}{6\pi\cdot 10}$cm. Since this number is also approximately $5.3$cm, we feel our result is reasonable.
%\begin{center}
%\begin{tikzpicture}
%\draw[thick] plot[domain=0:6.28, smooth] ({2*sin(\x r)},{-cos(\x r)+\x/10});
%\draw[thick] plot[domain=0:6.28, smooth] ({2*sin(\x r)},{-cos(\x r)+\x/10-.25});
%\draw[thick] (0,-1)--(0,-1.25) (0,-.372)--(0,-.622);
%\fill[opacity=0.25] (0,-1)--plot[domain=0:1.57, smooth] ({2*sin(\x r)},{-cos(\x r)+\x/10})  --
%plot[domain=1.57:0, smooth] ({2*sin(\x r)},{-cos(\x r)+\x/10-.25}) -- (0,-1.25)--(0,-1);
%\fill[opacity=0.25] (0,-.372)--plot[domain=6.28:4.71, smooth] ({2*sin(\x r)},{-cos(\x r)+\x/10})  --
%plot[domain=4.71:6.28, smooth] ({2*sin(\x r)},{-cos(\x r)+\x/10-.25}) -- (0,-.622)--(0,-.372);
%\draw (0,-2) node{Coil A};
%\end{tikzpicture}\hfill
%\begin{tikzpicture}
%\draw[thick] plot[domain=0:6.28, smooth] ({2*sin(\x r)},{-cos(\x r)});
%\draw[thick] plot[domain=0:6.28, smooth] ({2*sin(\x r)},{-cos(\x r)-.25});
%\draw[thick] (0,-1)--(0,-1.25);
%\fill[opacity=0.25] plot[domain=-1.57:1.57, smooth] ({2*sin(\x r)},{-cos(\x r)})-- plot[domain=1.57:-1.57, smooth] %({2*sin(\x r)},{-cos(\x r)-.25});
%\draw (0,-2) node{Coil B};
%\end{tikzpicture}\hfill
%\begin{tikzpicture}
%\draw[thick] plot[domain=0:6.28, smooth] ({2*sin(\x r)},{\x/2});
%\draw[thick] plot[domain=0:6.28, smooth] ({2*sin(\x r)},{\x/2-.25});
%\draw[thick] (0,0)--(0,-.25) (0,3.14)--(0,2.89);
%\fill[opacity=0.25] plot[domain=0:1.57, smooth] ({2*sin(\x r)},{\x/2})-- plot[domain=1.67:0, smooth] ({2*sin(\x r)},{\x/2-.25})--(0,0);
%\fill[opacity=0.25] plot[domain=4.71:6.28, smooth] ({2*sin(\x r)},{\x/2})-- plot[domain=6.28:4.71, smooth] ({2*sin(\x r)},{\x/2-.25});
%
%\draw (0,-1) node{Coil C};
%\end{tikzpicture}
%\end{center}



\end{solution}
%%%%%%%%%%%%%%%%%%
\begin{question}
A projectile falling under the influence of gravity and slowed
by air resistance proportional to its speed has position satisfying
\begin{equation*}
\difftwo{\vr}{t}=-g\hk-\alpha\diff{\vr}{t}
\end{equation*}
where $\alpha$ is a positive constant. If $\vr=\vr_0$ and 
$\diff{\vr}{t}=\vv_0$ at time $t=0$, find $\vr(t)$.
\end{question}

\begin{hint}
Define $\vu(t)=e^{\alpha t}\frac{d\vr}{dt}(t)$ and substitute
$\diff{\vr}{t}(t)=e^{-\alpha t}\vu(t)$ into the given 
differential equation to find a differential equation for $\vu$.
\end{hint}

\begin{answer}
$\vr(t)=\vr_0-\frac{e^{-\alpha t}-1}{\alpha}\vv_0
+g\frac{1-\alpha t-e^{-\alpha t}}{\alpha^2}\hk$
\end{answer}

\begin{solution}
Define $\vu(t)=e^{\alpha t}\frac{d\vr}{dt}(t)$. Then
\begin{align*}
\diff{\vu}{t}(t)&=\alpha e^{\alpha t}\diff{\vr}{t}(t)
+e^{\alpha t}\frac{d^2\vr}{dt^2}(t)\\
&=\alpha e^{\alpha t}\diff{\vr}{t}(t)
-ge^{\alpha t}\hk
-\alpha e^{\alpha t}\diff{\vr}{t}(t)\\
&=-ge^{\alpha t}\hk
\end{align*}
Integrating both sides of this equation from $t=0$ to $t=T$ gives
\begin{equation*}
\vu(T)-\vu(0)=-g\frac{e^{\alpha T}-1}{\alpha}\hk
\end{equation*}
so that
\begin{equation*}
\vu(T)=\vu(0)-g\frac{e^{\alpha T}-1}{\alpha}\hk
=\diff{\vr}{t}(0)-g\frac{e^{\alpha T}-1}{\alpha}\hk
=\vv_0-g\frac{e^{\alpha T}-1}{\alpha}\hk
\end{equation*}
Substituting in $\vu(T)=e^{\alpha t}\frac{d\vr}{dt}(T)$ and multiplying
through by $e^{-\alpha T}$ gives
\begin{equation*}
\diff{\vr}{t}(T)
=e^{-\alpha T}\vv_0-g\frac{1-e^{-\alpha T}}{\alpha}\hk
\end{equation*}
Integrating both sides of this equation from $T=0$ to $T=t$ gives
\begin{align*}
\vr(t)-\vr(0)
=\frac{e^{-\alpha t}-1}{-\alpha}\vv_0-g\frac{ t}{\alpha}\hk
+g\frac{e^{-\alpha t}-1}{-\alpha^2}\hk
\end{align*}
so that
\begin{align*}
\vr(t)=\vr_0-\frac{e^{-\alpha t}-1}{\alpha}\vv_0
+g\frac{1-\alpha t-e^{-\alpha t}}{\alpha^2}\hk
\end{align*}
\end{solution}

