\section*{Introduction}
First of all, welcome back to Calculus!

This book is an early draft of a companion question book for the \href{http://www.math.ubc.ca/~CLP/CLP4/clp_4_vc.pdf}{CLP-4 text}.
Additional questions are still under active development.

\subsection*{How to Work Questions}

This book is organized into four sections: Questions, Hints, Answers, and Solutions. As you are working problems, resist the temptation to prematurely peek at the back! It's important to allow yourself to struggle for a time with the material. Even professional mathematicians don't always know right away how to solve a problem. The art is in gathering your thoughts and figuring out a strategy to use what you know to find out what you don't.

If you find yourself at a real impasse, go ahead and look for a hint in the Hints section. Think about it for a while, and don't be afraid to read back in the notes to look for a key idea that will help you proceed. If you still can't solve the problem, well, we included the Solutions section for a reason! As you're reading the solutions, try hard to understand why we took the steps we did, instead of memorizing step-by-step how to solve that one particular problem.

If you struggled with a question quite a lot, it's probably a good idea to return to it in a few days. That might have been enough time for you to internalize the necessary ideas, and you might find it easily conquerable. Pat yourself on the back-sometimes math makes you feel good! If you're still having troubles, read over the solution again, with an emphasis on understanding why each step makes sense.

One of the reasons so many students are required to study calculus is the hope that it will improve their problem-solving skills. In this class, you will learn lots of concepts, and be asked to apply them in a variety of situations. Often, this will involve answering one really big problem by breaking it up into manageable chunks, solving those chunks, then putting the pieces back together. When you see a particularly long question, remain calm and look for a way to break it into pieces you can handle.

\subsection*{Working with Friends}

Study buddies are fantastic! If you don't already have friends in your class, you can ask your neighbours in lecture to form a group. Often, a question that you might bang your head against for an hour can be easily cleared up by a friend who sees what you've missed. Regular study times make sure you don't procrastinate too much, and friends help you maintain a positive attitude when you might otherwise succumb to frustration. Struggle in mathematics is desirable, but suffering is not.

When working in a group, make sure you try out problems on your own before coming together to discuss with others. Learning is a process, and getting answers to questions that you haven't considered on your own can rob you of the practice you need to master skills and concepts, and the tenacity you need to develop to become a competent problem-solver.

\subsection*{Types of Questions}
%\begin{Mquestion}%make this an Mquestion to be cute
%Questions outlined in blue make up the \emph{representative question set}. This set of questions is intended to cover the most essential ideas in each section. These questions are usually highly typical of what you'd see on an exam, although some of them are atypical but carry an important moral. If you find yourself unconfident with the idea behind one of these, it's probably a good idea to practice similar questions.
%
%This representative question set is our suggestion for a minimal selection of questions to work on.  You are highly encouraged to work on more.
%\end{Mquestion}

\begin{question}[year]
In addition to original problems, this book contains problems pulled from quizzes and exams given at UBC for Math 317 (Calculus 4). These problems are marked with a star. The authors would like to acknowledge the contributions of the many people who collaborated to produce these exams over the years.
\end{question}

%\Instructions{Instructions and other comments that are attached to more than one question are written in this font.}

The questions are organized into \Conceptual, \Procedural, and \Application. 

\subsection*{\Conceptual}The first category is meant to test and improve your understanding of basic underlying concepts. These often do not involve much calculation. They range in difficulty from very basic reviews of definitions to questions that require you to be thoughtful about the concepts covered in the section.

\subsection*{\Procedural} Questions in this category are for practicing skills. It's not enough to understand the philosophical grounding of an idea: you have to be able to apply it in appropriate situations. This takes practice!

\subsection*{\Application} The last questions in each section go a little farther than \Procedural. Often they will combine more than one idea, incorporate review material, or ask you to apply your understanding of a concept to a new situation.

In exams, as in life, you will encounter questions of varying difficulty. A good skill to practice is recognizing the level of difficulty a problem poses. Exams will have some easy questions, some standard questions, and some harder questions.

% \subsection*{Acknowledgements}
% Territorial acknowledgement
% 
% Elyse would like to thank her husband Seckin Demirbas for his endless patience, tireless support, and thoughtful feedback.
% 
% Joel and Andrew would like to thank some people, too.
