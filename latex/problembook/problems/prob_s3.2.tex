%\documentclass[12pt]{article}
\newcommand{\vt}{\mathbf{t}}
\newcommand{\vd}{\mathbf{d}}

\questionheader{ex:s3.2}

%%%%%%%%%%%%%%%%%%
\subsection*{\Conceptual}
%%%%%%%%%%%%%%%%%%

%%tangent planes
%\begin{question}
%A flagpole is sticking straight out of the surface blah. What direction is it pointing?
%\end{question}
%\begin{question}
%Sketch a tangent plane.
%\end{question}


%%%%%%%%%%%%%%%%%%%%%%%%%%%%%%%%
\begin{question}
Is it reasonable to say that the surfaces $x^2+y^2+(z-1)^2=1$ and
$x^2+y^2+(z+1)^2=1$ are tangent to each other at $(0,0,0)$?
\end{question}

\begin{hint}
What are the tangent planes to the two surfaces at $(0,0,0)$?
\end{hint}

\begin{answer}
Yes. The plane $z=0$ is the tangent plane to both surfaces at $(0,0,0)$.
\end{answer}

\begin{solution}
Write $F(x,y,z) = x^2+y^2+(z-1)^2-1$ and $G(x,y,z) = x^2+y^2+(z+1)^2-1$.
Let $S_1$ denote the surface $F(x,y,z)=0$ and $S_2$ denote the surface 
$G(x,y,z)=0$.
First note that $F(0,0,0)=G(0,0,0)=0$ so that the point $(0,0,0)$ lies
on both $S_1$ and $S_2$. The gradients of $F$ and $G$ are 
\begin{align*}
\vnabla F(x,y,z)
  &=\left(\pdiff{F}{x}(x,y,z)\,,\,
        \pdiff{F}{y}(x,y,z)\,,\,
        \pdiff{F}{z}(x,y,z)\right) 
    =\left( 2x\,,\,2y\,,\,2(z-1)\right) \\
\vnabla G(x,y,z)
  &=\left(\pdiff{G}{x}(x,y,z)\,,\,
        \pdiff{G}{y}(x,y,z)\,,\,
        \pdiff{G}{z}(x,y,z)\right) 
    =\left( 2x\,,\,2y\,,\,2(z+1)\right) 
\end{align*}
In particular,
\begin{equation*}
\vnabla F(0,0,0)=\left( 0,0,-2\right)\qquad
\vnabla G(0,0,0)=\left( 0,0,2\right)
\end{equation*}
so that the vector $\hk=-\frac{1}{2}\vnabla F(0,0,0)
                       =\frac{1}{2}\vnabla G(0,0,0)$
is normal to both surfaces at $(0,0,0)$. So the tangent plane to 
both $S_1$ and $S_2$ at $(0,0,0)$ is
\begin{equation*}
\hk\cdot\left(x-0,y-0,z-0\right)=0\qquad\text{or}\qquad z=0
\end{equation*}
Denote by $P$ the plane $z=0$. 
Thus $S_1$ is tangent to $P$ at $(0,0,0)$ and $P$ is tangent to $S_2$ 
at $(0,0,0)$. So it is reasonable to say that $S_1$ and $S_2$ are tangent at
$(0,0,0)$.


\end{solution}


%%%%%%%%%%%%%%%%%%%%%%%%%%%%%
\begin{question}\label{tan_line_plane}
Let the point $\vr_0= (x_0,y_0,z_0)$ lie on the surface $G(x,y,z)=0$.
Assume that $\vnabla G(x_0,y_0,z_0)\ne\vZero$. Suppose that the 
parametrized curve $\vr(t)=\big(x(t),y(t),z(t)\big)$ is contained in the surface
and that $\vr(t_0)=\vr_0$. Show that the tangent line to the curve at $\vr_0$
lies in the tangent plane to $G=0$ at $\vr_0$.

\end{question}

\begin{hint}
Apply the chain rule to $G\big(\vr(t)\big)=0$.
\end{hint}

\begin{answer}
See the solution.
\end{answer}

\begin{solution} 
Denote by $S$ the surface $G(x,y,z)=0$ and by $C$ the parametrized curve 
$\vr(t)=\big(x(t),y(t),z(t)\big)$. To start, we'll find the tangent plane to $S$ at $\vr_0$ and the tangent line to $C$ at $\vr_0$.  
\begin{itemize}
\item
The tangent vector to $C$ at $\vr_0$ is 
$\left( x'(t_0)\,,\,y'(t_0)\,,\,z'(t_0) \right)$, so the parametric equations for
the tangent line to $C$ at $\vr_0$ are
\begin{equation*}
x-x_0 = t x'(t_0)\qquad
y-y_0 = t x'(t_0)\qquad
z-z_0 = t x'(t_0)
\tag{$E_1$}\end{equation*}

\item
The gradient 
$\left(\pdiff{G}{x}\big( x_0\,,\,y_0\,,\,z_0\big)\,,\,
\pdiff{G}{y}\big( x_0\,,\,y_0\,,\,z_0\big)\,,\,
\pdiff{G}{z}\big( x_0\,,\,y_0\,,\,z_0\big)\right)$ is a normal vector 
to the surface $S$ at $(x_0,y_0,z_0)$. So the tangent plane
to the surface $S$ at $(x_0,y_0,z_0)$ is
\begin{equation*}
\left(\pdiff{G}{x}\big( x_0\,,\,y_0\,,\,z_0\big)\,,\,
\pdiff{G}{y}\big( x_0\,,\,y_0\,,\,z_0\big)\,,\,
\pdiff{G}{z}\big( x_0\,,\,y_0\,,\,z_0\big)\right) \cdot
\left( x-x_0\,,\, y-y_0\,,\,z-z_0\right) = 0
\end{equation*}
or
\begin{equation*}
\pdiff{G}{x}\big( x_0\,,\,y_0\,,\,z_0\big)\ (x-x_0)
+\pdiff{G}{y}\big( x_0\,,\,y_0\,,\,z_0\big)\ (y-y_0)
+\pdiff{G}{z}\big( x_0\,,\,y_0\,,\,z_0\big)\ (z-z_0) = 0
\tag{$E_2$}\end{equation*}

\end{itemize}
Next, we'll show that the tangent vector  
$\left( x'(t_0)\,,\,y'(t_0)\,,\,z'(t_0) \right)$ to $C$ at $\vr_0$ and the normal vector $\left(\pdiff{G}{x}\big( x_0\,,\,y_0\,,\,z_0\big)\,,\,
\pdiff{G}{y}\big( x_0\,,\,y_0\,,\,z_0\big)\,,\,
\pdiff{G}{z}\big( x_0\,,\,y_0\,,\,z_0\big)\right)$ to $S$ at $\vr_0$ are 
perpendicular to each other. To do so, we observe that,
for every $t$, the point $\big(x(t),y(t),z(t)\big)$
lies on the surface $G(x,y,z)=0$ and so obeys
\begin{align*}
G\big(x(t),y(t),z(t)\big) =0
\end{align*}
Differentiating this equation with respect to $t$ gives,
by the chain rule,
\begin{align*}
0&= \diff{}{t}G\big(x(t),y(t),z(t)\big) \\
&=\pdiff{G}{x}\big( x(t)\,,\,y(t)\,,\,z(t)\big)\ x'(t)
+\pdiff{G}{y}\big( x(t)\,,\,y(t)\,,\,z(t)\big)\ y'(t)
+\pdiff{G}{z}\big( x(t)\,,\,y(t)\,,\,z(t)\big)\ z'(t)
\end{align*}
Then setting $t=t_0$ gives 
\begin{equation*}
\pdiff{G}{x}\big( x_0\,,\,y_0\,,\,z_0\big)\ x'(t_0)
+\pdiff{G}{y}\big( x_0\,,\,y_0\,,\,z_0\big)\ y'(t_0)
+\pdiff{G}{z}\big( x_0\,,\,y_0\,,\,z_0\big)\ z'(t_0) = 0
\tag{$E_3$}
\end{equation*}
Finally, we are in a position to show that if $(x,y,z)$ is any point on 
the tangent line to $C$ at $\vr_0$, then $(x,y,z)$ is also on the 
tangent plane to $S$ at $\vr_0$. As $(x,y,z)$ is on the tangent line to $C$ 
at $\vr_0$ then there is a $t$ such that, by $(E_1)$,
\begin{align*}
&\pdiff{G}{x}\big( x_0\,,\,y_0\,,\,z_0\big)\ \textcolor{blue}{\{x-x_0\}}
+\pdiff{G}{y}\big( x_0\,,\,y_0\,,\,z_0\big)\ \textcolor{blue}{\{y-y_0\}}
+\pdiff{G}{z}\big( x_0\,,\,y_0\,,\,z_0\big)\ \textcolor{blue}{\{z-z_0\}}
\\
&=\pdiff{G}{x}\big( x_0\,,\,y_0\,,\,z_0\big)\ 
               \textcolor{blue}{\big\{ t\,x'(t_0)\big\}}
+\pdiff{G}{y}\big( x_0\,,\,y_0\,,\,z_0\big)\ 
               \textcolor{blue}{\big\{ t\,y'(t_0)\big\}}
+\pdiff{G}{z}\big( x_0\,,\,y_0\,,\,z_0\big)\ 
               \textcolor{blue}{\big\{ t\,z'(t_0)\big\}}\\
&=\textcolor{blue}{t}\left[\pdiff{G}{x}\big( x_0\,,\,y_0\,,\,z_0\big)\ 
                \textcolor{blue}{x'(t_0)}
+\pdiff{G}{y}\big( x_0\,,\,y_0\,,\,z_0\big)\ 
                \textcolor{blue}{y'(t_0)}
+\pdiff{G}{z}\big( x_0\,,\,y_0\,,\,z_0\big)\ 
                 \textcolor{blue}{z'(t_0)} \right]
=0
\end{align*}
by $(E_3)$. That is, $(x,y,z)$ obeys the equation, $(E_2)$, of the tangent plane to $S$ at $\vr_0$ and so is on that tangent plane.  So the tangent 
line to $C$ at $\vr_0$ is contained in the tangent plane to $S$ at $\vr_0$.
\end{solution}

%%%%%%%%%%%%%%%%%%%%%%%%%%%%%%%%
\begin{question}
Find the parametric equations of the normal line to the surface 
$z=f(x,y)$ at the point $\big(x_0\,,\,y_0\,,\,z_0\!=\!f(x_0,y_0)\big)$. 
By definition, the normal line in question is the line through $(x_0,y_0,z_0)$ 
whose direction vector is perpendicular to the surface at $(x_0,y_0,z_0)$.
\end{question}

%\begin{hint}
%\end{hint}

\begin{answer}
\begin{align*}
&(x-x_0\,,\,y-y_0\,,\,z-z_0) 
      = t\big(-f_x(x_0,y_0)\,,\,- f_y(x_0,y_0)\,,\, 1\big) 
\qquad\text{or}\\
&x=x_0-tf_x(x_0,y_0)\quad
y=y_0-tf_y(x_0,y_0)\quad
z=f(x_0,y_0)+t
\end{align*}
\end{answer}

\begin{solution}
By part (b) of Theorem \eref{CLP317}{thm:normalVectors} in the CLP-4 text,
\begin{align*}
\vn =-f_x(x_0,y_0)\,\hi - f_y(x_0,y_0)\,\hj + \hk
\end{align*}
is normal to the surface at $(x_0,y_0,z_0)$. So the parametric equations of the
normal line are
\begin{align*}
&(x-x_0\,,\,y-y_0\,,\,z-z_0) 
      = t\big(-f_x(x_0,y_0)\,,\,- f_y(x_0,y_0)\,,\, 1\big) 
\qquad\text{or}\\
&x=x_0-tf_x(x_0,y_0)\quad
y=y_0-tf_y(x_0,y_0)\quad
z=f(x_0,y_0)+t
\end{align*}
\end{solution}

%%%%%%%%%%%%%%%%%%%%%%%%%%%%%%%%
\begin{question}
Let $F(x_0,y_0,z_0)=G(x_0,y_0,z_0)=0$ and let the vectors
$\vnabla F(x_0,y_0,z_0)$ and $\vnabla G(x_0,y_0,z_0)$ be nonzero and not
be parallel to each other. Find the equation of the normal plane to the 
curve of intersection of the surfaces $F(x,y,z)=0$ and $G(x,y,z)=0$ at
$(x_0,y_0,z_0)$. By definition, that normal plane is the plane through
$(x_0,y_0,z_0)$ whose normal vector is the tangent vector to the curve of
intersection at $(x_0,y_0,z_0)$. 
\end{question}

\begin{hint}
To find a tangent vector to the curve of intersection of the 
surfaces $F(x,y,z)=0$ and $G(x,y,z)=0$ at $(x_0,y_0,z_0)$,
use Q[\ref{tan_line_plane}] twice, once for the surface $F(x,y,z)=0$ and
once for the surface $G(x,y,z)=0$.
\end{hint}

\begin{answer}
The normal plane is $\vn\cdot\left( x-x_0\,,\,y-y_0\,,\,z-z_0\right) =0$,
where the normal vector 
$\vn = \vnabla F(x_0,y_0,z_0)\times \vnabla G(x_0,y_0,z_0)$.
\end{answer}

\begin{solution}
Use $S_1$ to denote the surface $F(x,y,z)=0$, 
    $S_2$ to denote the surface $G(x,y,z)=0$ and 
    $C$ to denote the curve of intersection of $S_1$ and $S_2$.
\begin{itemize}
\item
Since $C$ is contained in $S_1$, the tangent line to $C$ at $(x_0,y_0,z_0)$
is contained in the tangent plane to $S_1$ at $(x_0,y_0,z_0)$, by 
Q[\ref{tan_line_plane}]. In particular, any tangent vector, $\vt$, to 
$C$ at $(x_0,y_0,z_0)$ must be perpendicular to $\vnabla F(x_0,y_0,z_0)$,
the normal vector to $S_1$ at $(x_0,y_0,z_0)$.

\item
Since $C$ is contained in $S_2$, the tangent line to $C$ at $(x_0,y_0,z_0)$
is contained in the tangent plane to $S_2$ at $(x_0,y_0,z_0)$, by 
Q[\ref{tan_line_plane}]. In particular, any tangent vector, $\vt$, to 
$C$ at $(x_0,y_0,z_0)$ must be perpendicular to $\vnabla G(x_0,y_0,z_0)$,
the normal vector to $S_2$ at $(x_0,y_0,z_0)$.
\end{itemize}
So any tangent vector to $C$ at $(x_0,y_0,z_0)$ must be perpendiular to both
$\vnabla F(x_0,y_0,z_0)$ and $\vnabla G(x_0,y_0,z_0)$.
One such tangent vector is
\begin{align*}
\vt = \vnabla F(x_0,y_0,z_0)\times \vnabla G(x_0,y_0,z_0)
\end{align*}
(Because the vectors $\vnabla F(x_0,y_0,z_0)$ and $\vnabla G(x_0,y_0,z_0)$
are nonzero and not parallel, $\vt$ is nonzero.) So the normal plane in 
question passes through $(x_0,y_0,z_0)$ and has normal vector $\vn=\vt$.
Consquently, the normal plane is
\begin{equation*}
\vn\cdot\left( x-x_0\,,\,y-y_0\,,\,z-z_0\right) =0 \qquad\text{where }
\vn=\vt=\vnabla F(x_0,y_0,z_0)\times \vnabla G(x_0,y_0,z_0)
\end{equation*}


\end{solution}

%%%%%%%%%%%%%%%%%%%%%%%%%%%%%%%%
\begin{question}
Let $f(x_0,y_0)=g(x_0,y_0)$ and let 
$\left( f_x(x_0,y_0), f_y(x_0,y_0)\right)\ne \left( g_x(x_0,y_0), g_y(x_0,y_0)\right)$. Find the equation of the tangent line to the 
curve of intersection of the surfaces $z=f(x,y)$ and $z=g(x,y)$ at
$(x_0\,,\,y_0\,,\,z_0=f(x_0,y_0))$.
\end{question}

\begin{hint}
To find a tangent vector to the curve of intersection of the 
surfaces $z=f(x,y)$ and $z=g(x,y)$ at $(x_0,y_0,z_0)$,
use Q[\ref{tan_line_plane}] twice, once for the surface $z=f(x,y)$ and
once for the surface $z=g(x,y)$.
\end{hint}

\begin{answer}
Tangent line is
\begin{align*}
x&=x_0+t\big[g_y(x_0,y_0)-f_y(x_0,y_0)\big] \\
y&=y_0+t\big[f_x(x_0,y_0)-g_x(x_0,y_0)\big] \\
z&=z_0+ t\big[f_x(x_0,y_0)g_y(x_0,y_0)-f_y(x_0,y_0)g_x(x_0,y_0)\big]
\end{align*}
\end{answer}

\begin{solution}
Use $S_1$ to denote the surface $z=f(x,y)$, 
    $S_2$ to denote the surface $z=g(x,y)$ and 
    $C$ to denote the curve of intersection of $S_1$ and $S_2$.
\begin{itemize}
\item
Since $C$ is contained in $S_1$, the tangent line to $C$ at $(x_0,y_0,z_0)$
is contained in the tangent plane to $S_1$ at $(x_0,y_0,z_0)$, by 
Q[\ref{tan_line_plane}]. In particular, any tangent vector, $\vt$, to 
$C$ at $(x_0,y_0,z_0)$ must be perpendicular to $-f_x(x_0,y_0)\,\hi
-f_y(x_0,y_0)\,\hj+\hk$,
the normal vector to $S_1$ at $(x_0,y_0,z_0)$.
(See part (b) of Theorem \eref{CLP317}{thm:normalVectors} in the CLP-4 text.)

\item
Since $C$ is contained in $S_2$, the tangent line to $C$ at $(x_0,y_0,z_0)$
is contained in the tangent plane to $S_2$ at $(x_0,y_0,z_0)$, by 
Q[\ref{tan_line_plane}]. In particular, any tangent vector, $\vt$, to 
$C$ at $(x_0,y_0,z_0)$ must be perpendicular to $-g_x(x_0,y_0)\,\hi
-g_y(x_0,y_0)\,\hj+\hk$,
the normal vector to $S_2$ at $(x_0,y_0,z_0)$.
\end{itemize}
So any tangent vector to $C$ at $(x_0,y_0,z_0)$ must be perpendicular to
both of the vectors $-f_x(x_0,y_0)\,\hi-f_y(x_0,y_0)\,\hj+\hk$ and
$-g_x(x_0,y_0)\,\hi -g_y(x_0,y_0)\,\hj+\hk$.
One such tangent vector is
\begin{align*}
&\vt = \big[-f_x(x_0,y_0)\,\hi - f_y(x_0,y_0)\,\hj+\hk\big]\times 
       \big[-g_x(x_0,y_0)\,\hi - g_y(x_0,y_0)\,\hj+\hk\big] \\
    &=\det\left[\begin{matrix}
                     \hi &        \hj &   \hk \\
           -f_x(x_0,y_0) & -f_y(x_0,y_0) & 1 \\
           -g_x(x_0,y_0) & -g_y(x_0,y_0) & 1
                \end{matrix}\right] \\
    &=\left( g_y(x_0,y_0)-f_y(x_0,y_0)\,,\,
           f_x(x_0,y_0)-g_x(x_0,y_0)\,,\,
           f_x(x_0,y_0)g_y(x_0,y_0)-f_y(x_0,y_0)g_x(x_0,y_0)\right)
\end{align*}
So the tangent line in question passes through $(x_0,y_0,z_0)$ and has 
direction vector $\vd=\vt$. Consquently, the tangent line is 
\begin{equation*}
\left( x-x_0\,,\,y-y_0\,,\,z-z_0\right) = t\,\vd
\end{equation*}
or
\begin{align*}
x&=x_0+t\big[g_y(x_0,y_0)-f_y(x_0,y_0)\big] \\
y&=y_0+t\big[f_x(x_0,y_0)-g_x(x_0,y_0)\big] \\
z&=z_0+ t\big[f_x(x_0,y_0)g_y(x_0,y_0)-f_y(x_0,y_0)g_x(x_0,y_0)\big]
\end{align*}
\end{solution}




%%%%%%%%%%%%%%%%%%
\subsection*{\Procedural}
%%%%%%%%%%%%%%%%%%

%%%%%%%%%%%%%%%%%%%%%%%%%%%%%%%%
\begin{question}[M200 2009A] %1a
Let $\displaystyle f(x,y)=\frac{x^2y}{x^4+2y^2}$.
Find the tangent plane to the surface $z = f(x,y)$ at the point
$\left( -1\,,\,1\,,\,\frac{1}{3}\right)$.
\end{question}

%\begin{hint}
%
%\end{hint}

\begin{answer}
$2x+y+9z=2$
\end{answer}

\begin{solution}
We are going to use part (b) of Theorem \eref{CLP317}{thm:normalVectors} 
in the CLP-4 text. To do so, we need the first order derivatives of $f(x,y)$ 
at $(x,y)=(-1,1)$. So we find them first.
\begin{alignat*}{3}
f_x(x,y)&=\frac{2xy}{x^4+2y^2}-\frac{x^2y(4x^3)}{{(x^4+2y^2)}^2}\qquad &
f_x(-1,1)&=-\frac{2}{3} +\frac{4}{3^2}=-\frac{2}{9}
\\
f_y(x,y)&=\frac{x^2}{x^4+2y^2}-\frac{x^2y(4y)}{{(x^4+2y^2)}^2}\qquad &
f_y(-1,1)&=\frac{1}{3} -\frac{4}{3^2}=-\frac{1}{9}
\end{alignat*}
So $(2/9\,,\,1/9\,,\,1)$ is a normal vector to the surface at 
$(-1,1,1/3)$ and the tangent plane is
\begin{align*}
\frac{2}{9}(x+1) +\frac{1}{9}(y-1) +\Big(z-\frac{1}{3}\Big) &=0 \\
\frac{2}{9}x +\frac{1}{9}y + z &=-\frac{2}{9}+\frac{1}{9}+\frac{1}{3}
           =\frac{2}{9}  
\end{align*}
or $2x+y+9z=2$.
\end{solution}

%%%%%%%%%%%%%%%%%%%%%%%%%%%%%%%%
\begin{question}[M200 2015D] %1c
Find the tangent plane to
\begin{equation*}
\frac{27}{\sqrt{x^2+y^2+z^2+3}}=9
\end{equation*}
at the point $(2, 1, 1)$.
\end{question}

%\begin{hint}
%
%\end{hint}

\begin{answer}
$2x+y+z = 6$
\end{answer}

\begin{solution}
The equation of the given surface is of the form $G(x,y,z)=9$
with $G(x,y,z) =\frac{27}{\sqrt{x^2+y^2+z^2+3}}$. So,
by part (c) of Theorem \eref{CLP317}{thm:normalVectors} 
in the CLP-4 text, a normal
vector to the surface at $(2,1,1)$ is
\begin{align*}
\vnabla G(2,1,1)
  &=-\frac{1}{2}\ \frac{27}{(x^2+y^2+z^2+3)^{3/2}}\big(2x\,,\,2y\,,\,2z\big)
                                          \bigg|_{(x,y,z)=(2,1,1)} \\
  &=-( 2\,,\,1\,,\,1)
\end{align*}
and the equation of the tangent plane is
\begin{equation*}
-( 2\,,\,1\,,\,1)\cdot ( x-2\,,\,y-1\,,\,z-1)=0\qquad\text{or}\qquad
2x+y+z = 6
\end{equation*}
\end{solution}

%%%%%%%%%%%%%%%%%%%%%%%%%%%%%%%%
\begin{question}[M200 2005D] %4
Consider the surface $z = f(x,y)$ defined implicitly by the equation 
$xyz^2 + y^2 z^3 = 3 + x^2$. Use a 3--dimensional gradient vector 
to find the equation of the tangent plane to this surface at the point
$(-1, 1, 2)$. Write your answer in the form $z = ax + by + c$, where 
$a$, $b$ and $c$ are constants.
\end{question}

%\begin{hint}
%
%\end{hint}

\begin{answer}
$z = -\frac{3}{4} x- \frac{3}{2} y + \frac{11}{4}$
\end{answer}

\begin{solution}
We may use $G(x,y,z) = xyz^2 + y^2 z^3 - 3 - x^2 = 0$ as an equation for
the surface.  Note that $(-1,1,2)$ really is on the surface since
\begin{align*}
G(-1,1,2) = (-1)(1)(2)^2 + (1)^2 (2)^3 - 3 - (-1)^2 
          = -4 + 8 - 3 - 1
          =0
\end{align*}
By part (c) of Theorem \eref{CLP317}{thm:normalVectors} 
in the CLP-4 text, since
\begin{alignat*}{5}
G_x(x,y,z)&=yz^2 -2x \qquad & 
    G_x(-1,1,2)&=6  \\
G_y(x,y,z)&=xz^2 +2yz^3 \qquad & 
    G_y(-1,1,2)&=12  \\
G_z(x,y,z)&=2xyz+3y^2z^2 \qquad & 
    G_z(-1,1,2)&=8  
\end{alignat*}
one normal vector to the surface at $(-1,1,2)$ is 
 $\vnabla G(-1,1,2) = ( 6\,,\,12\,,\,8)$ and an equation
of the tangent plane to the surface at $(-1,1,2)$ is
\begin{align*}
( 6\,,\,12\,,\,8) \cdot
     ( x+1\,,\,y-1\,,\,z-2) = 0\qquad\text{or}\qquad
6x+12 y+ 8z = 22
\end{align*}
or
\begin{equation*}
z = -\frac{3}{4} x- \frac{3}{2} y +\frac{11}{4}
\end{equation*}
\end{solution}

%%%%%%%%%%%%%%%%%%%%%%%%%%%%%%%%
\begin{question}[M200 2008D] %1
A surface is given by
\begin{equation*}
z = x^2 - 2xy + y^2 .
\end{equation*}

\begin{enumerate}[(a)]
\item
Find the equation of the tangent plane to the surface at $x = a$, $y = 2a$.

\item 
For what value of $a$ is the tangent plane parallel to the plane 
$x - y + z = 1$?
\end{enumerate}
\end{question}

%\begin{hint}
%
%\end{hint}

\begin{answer}
(a) $2ax -2ay +z = -a^2$\qquad
(b) $a=\frac{1}{2}$.
\end{answer}

\begin{solution}
(a)
The surface is $G(x,y,z)=z-x^2+2xy-y^2=0$. When $x=a$ and $y=2a$
 and $(x,y,z)$ is on the surface, we have $z= a^2-2(a)(2a) +(2a)^2=a^2$.
So, by part (c) of Theorem \eref{CLP317}{thm:normalVectors} 
in the CLP-4 text, 
a normal vector to this surface at $(a,2a,a^2)$ is
\begin{align*}
\vnabla G(a,2a,a^2) = ( -2x+2y\,,\,2x-2y\,,\,1)\Big|_{(x,y,z)=(a,2a,a^2)}
                    = ( 2a\,,\,-2a\,,\,1)
\end{align*}
and the equation of the tangent plane is 
\begin{align*}
( 2a\,,\,-2a\,,\,1)\cdot( x-a\,,\,y-2a\,,\,z-a^2) =0
\qquad\text{or}\qquad
2ax -2ay +z = -a^2
\end{align*}

(b) The two planes are parallel when their two normal vectors,
namely $( 2a\,,\,-2a\,,\,1)$ and $( 1\,,\,-1\,,\,1)$,
are parallel. This is the case if and only if $a=\frac{1}{2}$.
\end{solution}


%%%%%%%%%%%%%%%%%%%%%%%%%%%
\begin{question}[M317 2008D] %7
A surface S is given by the parametric equations
\begin{align*}
x &= 2u^2 \\
y &= v^2 \\
z &= u^2 + v^3
\end{align*}
Find an equation for the tangent plane to $S$ at the point $(8, 1, 5)$.
\end{question}

\begin{hint} 
Review \S\eref{CLP317}{sec:tangentPlanes} in the CLP-4 text.
\end{hint}

\begin{answer} 
$x+3y-2z = 1$
\end{answer}

\begin{solution} 
A plane is determined by one point on the plane and one vector 
perpendicular to the plane. We are told that $(8,1,5)$ is on the plane,
so it suffices to find a normal vector. The given surface is parametrized by
\begin{align*}
\vr(u,v) = 2u^2\,\hi + v^2\,\hj +(u^2+v^3)\,\hk
\end{align*}
so the vectors 
\begin{align*}
\frac{\partial \vr}{\partial u}(u,v)
 &= \big(4u\,,\, 0 \,,\, 2u\big) \\
\frac{\partial \vr}{\partial v}(u,v)
 &= \big(0\,,\, 2v \,,\, 3v^2\big) 
\end{align*}
are tangent to $S$ at $\vr(u,v)$. Note that $\vr(2,1)=(8,1,5)$.
So 
\begin{align*}
\frac{\partial \vr}{\partial u}(2,1)
 &= \big(8\,,\, 0 \,,\, 4\big) \\
\frac{\partial \vr}{\partial v}(2,1)
 &= \big(0\,,\, 2\,,\, 3\big) 
\end{align*}
are tangent to $S$ at $\vr(2,1)=(8,1,5)$ and
\begin{align*}
\frac{\partial \vr}{\partial u}(2,1) \times
     \frac{\partial \vr}{\partial v}(2,1)
&=\big(8\,,\, 0 \,,\, 4\big)\times \big(0\,,\, 2\,,\, 3\big)  \\
&=\det\left[\begin{matrix}
                      \hi & \hj & \hk \\
                      8   & 0  &  4  \\
                      0   & 2  &  3 \end{matrix} \right] \\ 
&= \big(-8\,,\,-24\,,\,16)
\end{align*}
or $\frac{1}{-8}\big(-8\,,\,-24\,,\,16) = \big(1\,,\,3\,,\,-2)$
is normal to $S$ at $(8,1,5)$. So the tangent plane is
\begin{align*}
(1,3,-2)\cdot\big\{(x,y,z)-(8,1,5)\big\}=0\qquad\text{or}\qquad
x+3y-2z = 1
\end{align*}


\end{solution}


\begin{question}[M317 2010D]   %3a
Let $S$ be the surface given by
\begin{equation*}
\vr(u, v) = \big( u + v\,,\, u^2 + v^2 \,,\, u - v\big),\qquad
-2 \le u \le 2,\ -2 \le v \le 2
\end{equation*}
Find the tangent plane to the surface at the point $(2, 2, 0)$.
\end{question}

\begin{hint} 
Review \S\eref{CLP317}{sec:tangentPlanes} in the CLP-4 text.
\end{hint}

\begin{answer} 
        $y=2x-2$
\end{answer}

\begin{solution} 
To find the tangent plane we have to find a normal vector 
to the surface at $(2,2,0)$. Since
\begin{align*}
\frac{\partial\vr}{\partial u}
&=\big(1\,,\,
       2u\,,\,
        1 \big) \\
\frac{\partial\vr}{\partial v}
&=\big(1\,,\,
       2v\,,\,
       -1 \big) 
\end{align*}
a normal vector to the surface at $\vr(u,v)$ is
\begin{align*}
\frac{\partial\vr}{\partial u} \times \frac{\partial\vr}{\partial v} 
&=\det\left[\begin{matrix}
                      \hi & \hj & \hk \\
                      1   & 2u  &  1  \\
                      1   & 2v  & -1 \end{matrix} \right] \\ 
&=\big(-2u-2v\,,\,
       2\,,\,
       2v-2u\big)
\end{align*}
As $\vr(u,v) = (2,2,0)$ when (the $x$-coordinate) $u+v=2$ and 
(the $z$-coordinate) $u-v=0$, i.e when $u=v=1$,
a normal vector to the surface at $(2,2,0)=\vr(1,1)$ is
\begin{align*}
(-4,2,0)\qquad\text{or}\qquad (-2,1,0)
\end{align*}
and the equation of the specified tangent plane is
\begin{align*}
-2(x-2) +(y-2) +0z= 0\qquad\text{or} \qquad y=2x-2
\end{align*}
\end{solution}

%%%%%%%%%%%%%%%%%%%%%%%%%%%%%%%%
\begin{question}[M200 2010D] %1b
Find the tangent plane and normal line to the surface 
$z=f(x,y)=\frac{2y}{x^2+y^2}$ at $(x,y)=(-1,2)$.
\end{question}

%\begin{hint}
%
%\end{hint}

\begin{answer}
The tangent plane is $\frac{8}{25}x-\frac{6}{25}y-z=-\frac{8}{5}$.\\ 
  \null\hskip0.3in   The normal line is 
   $( x,y,z) = ( -1,2,\frac{4}{5}) 
                  +t ( \frac{8}{25}\,,\,-\frac{6}{25}\,,\,-1)$.
\end{answer}

\begin{solution}
The first order partial derivatives of $f$ are
\begin{alignat*}{3}
f_x(x,y) & = -\frac{4xy}{{(x^2+y^2)}^2}\quad &
      f_x(-1,2) & = \frac{8}{25} \\
f_y(x,y) & = \frac{2}{x^2+y^2}-\frac{4y^2}{{(x^2+y^2)}^2}\quad &
      f_y(-1,2) & = \frac{2}{5}-\frac{16}{25}
                  =-\frac{6}{25} \\
\end{alignat*}
So, by part (b) of Theorem \eref{CLP317}{thm:normalVectors} 
in the CLP-4 text, 
a normal vector to the surface at $(x,y)=(-1,2)$ is
$( \frac{8}{25}\,,\,-\frac{6}{25}\,,\,-1)$.
As $f(-1,2)= \frac{4}{5}$, the tangent plane is
\begin{align*}
\Big( \frac{8}{25}\,,\,-\frac{6}{25}\,,\,-1\Big)\cdot
     \Big( x+1\,,\,y-2\,,\,z -\frac{4}{5}\Big)=0\quad \text{or}\quad
\frac{8}{25}x-\frac{6}{25}y-z=-\frac{8}{5} % =-\frac{40}{25}
\end{align*} 
and the normal line is
\begin{align*}
( x,y,z) = \Big( -1,2,\frac{4}{5}\Big) 
                  +t \Big( \frac{8}{25}\,,\,-\frac{6}{25}\,,\,-1\Big)
\end{align*}
\end{solution}

\begin{question}[M200 2013D] %1f
Find all the points on the surface $x^2 + 9y^2 + 4z^2 = 17$ 
where the tangent plane is parallel to the plane $x - 8z = 0$.
\end{question}

\begin{hint}
Let $(x,y,z)$ be a desired point. Then  
\begin{itemize}\itemsep1pt \parskip0pt \parsep0pt %\itemindent-15pt
\item 
$(x,y,z)$ must be on the surface and
\item
the normal vector to the surface at $(x,y,z)$ must be parallel to the
plane's normal vector.
\end{itemize}
\end{hint}

\begin{answer}
$\pm(1,0,-2)$
\end{answer}

\begin{solution}
A normal vector to the surface $x^2 + 9y^2 + 4z^2 = 17$
at the point $(x,y,z)$ is $( 2x\,,\, 18y\,,\,8z)$. 
A normal vector to the plane $x - 8z = 0$ is $( 1\,,\,0\,,\,-8)$.
So we want $( 2x\,,\, 18y\,,\,8z)$ to be parallel to
$( 1\,,\,0\,,\,-8)$, i.e. to be a nonzero constant times
$( 1\,,\,0\,,\,-8)$. This is the case whenever $y=0$ and $z=-2x$
with $x\ne 0$. In addition, we want $(x,y,z)$ to lie on the surface
$x^2 + 9y^2 + 4z^2 = 17$. So we want $y=0$, $z=-2x$ and
\begin{align*}
17= x^2 + 9y^2 + 4z^2 =x^2 +4(-2x)^2=17x^2
\implies x=\pm 1
\end{align*} 
So the allowed points are $\pm(1,0,-2)$.
\end{solution}

\begin{question}[M200 2014D] %4
Let $S$ be the surface $z = x^2 + 2y^2 + 2y - 1$. Find all points 
$P (x_0,y_0,z_0)$ on $S$ with $x_0 \ne 0$ such that the normal line 
at $P$ contains the origin $(0,0,0)$.
\end{question}

\begin{hint}
First find a parametric equation for the normal line to $S$ at $(x_0,y_0,z_0)$.
Then the requirement that $(0,0,0)$ lies on that normal line gives 
three equations in the four unknowns $x_0,y_0,z_0$ and $t$. The requirement
that $(x_0,y_0,z_0)$ lies on $S$ gives a fourth equation. Solve this system of four equations.
\end{hint}

\begin{answer}
$\big(\frac{1}{\sqrt{2}}\,,\,-1\,,\,-\frac{1}{2}\big)$
and 
  $\big(-\frac{1}{\sqrt{2}}\,,\,-1\,,\,-\frac{1}{2}\big)$
\end{answer}
\begin{solution}
The equation of $S$ is of the form $G(x,y,z) = x^2 + 2y^2 + 2y-z = 1$.
So one normal vector to $S$ at the point $(x_0,y_0,z_0)$ is 
\begin{equation*}
\vnabla G(x_0,y_0,z_0)  = 2x_0\,\hi + (4y_0+2)\,\hj -\hk
\end{equation*}
and the normal line to $S$ at $(x_0,y_0,z_0)$ is
\begin{equation*}
(x,y,z) = (x_0,y_0,z_0) +t( 2x_0\,,\,4y_0+2\,,\, -1)
\end{equation*}
For this normal line to pass through the origin, there must be a $t$
with
\begin{align*}
(0,0,0) = (x_0,y_0,z_0) +t( 2x_0\,,\,4y_0+2\,,\, -1)
\end{align*}
or
\begin{align*}
x_0 + 2x_0\,t & =0 \tag{E1}\\
y_0 +(4y_0+2)t &=0 \tag{E2}\\
z_0 -t &=0 \tag{E3}
\end{align*}
Equation (E3) forces $t=z_0$. Substituting this into equations (E1) and (E2)
gives
\begin{align*}
x_0(1+2z_0) & =0 \tag{E1}\\
y_0 +(4y_0+2)z_0 &=0 \tag{E2}
\end{align*}
The question specifies that $x_0\ne 0$, so (E1) forces $z_0=-\frac{1}{2}$.
Substituting $z_0=-\frac{1}{2}$ into (E2) gives
\begin{equation*}
-y_0-1=0 \implies y_0=-1
\end{equation*}
Finally $x_0$ is determined by the requirement that $(x_0,y_0,z_0)$
must lie on $S$ and so must obey
\begin{equation*}
z_0 = x_0^2 + 2y_0^2 + 2y_0 - 1
\implies -\frac{1}{2} = x_0^2 + 2(-1)^2 +2(-1)-1
\implies x_0^2 = \frac{1}{2}
%\implies x_0 = \pm \frac{1}{\sqrt{2}}
\end{equation*}
So the allowed points $P$ are 
  $\big(\frac{1}{\sqrt{2}}\,,\,-1\,,\,-\frac{1}{2}\big)$
and 
  $\big(-\frac{1}{\sqrt{2}}\,,\,-1\,,\,-\frac{1}{2}\big)$.
\end{solution}

%%%%%%%%%%%%%%%%%%%%%%%%%%%%%%%%
\begin{question}[M226 2009D] %1b
Find all points on the hyperboloid $z^2=4x^2+y^2-1$
where the tangent plane is parallel to the plane $2x-y+z=0$.
\end{question}

\begin{hint}
Two (nonzero) vectors $\vv$ and $\vw$ are parallel if and only if there  
is a $t$ such that $\vv=t\,\vw$.
Don't forget that the point has to be on the hyperboloid.
\end{hint}

\begin{answer}
$\pm \big(\half,-1,-1\big)$
\end{answer}

\begin{solution} 
Let $(x_0,y_0,z_0)$  be a point on the hyperboloid $z^2=4x^2+y^2-1$
where the tangent plane is parallel to the plane $2x-y+z=0$. A normal vector
to the plane $2x-y+z=0$ is $( 2,-1,1)$. Because the hyperboloid is
$G(x,y,z)=4x^2+y^2-z^2-1$ and $\vnabla G(x,y,z) = ( 8x,2y,-2z)$,
 a normal vector to the hyperboloid at $(x_0,y_0,z_0)$ is 
$\vnabla G(x_0,y_0,z_0)=( 8x_0,2y_0,-2z_0)$. 
So $(x_0,y_0,z_0)$ satisfies the required conditions if and only if there is a nonzero $t$ obeying
\begin{align*}
&( 8x_0,2y_0,-2z_0) =t( 2,-1,1) \text{ and }
 z_0^2=4x_0^2+y_0^2-1\\
&\iff x_0=\frac{t}{4},\ y_0=z_0=-\frac{t}{2}\text{ and }
 z_0^2=4x_0^2+y_0^2-1\\
&\iff \frac{t^2}{4}= \frac{t^2}{4}+ \frac{t^2}{4}-1\text{ and }
     x_0=\frac{t}{4},\ y_0=z_0=-\frac{t}{2}\\
& \iff t=\pm 2\qquad
(x_0,y_0,z_0)=\pm \big(\half,-1,-1\big)
\end{align*}

\end{solution}

%%%%%%%%%%%%%%%%%%
\subsection*{\Application}
%%%%%%%%%%%%%%%%%%

%%%%%%%%%%%%%%%%%%%%%%%%%%%%%%%%
\begin{question}[M200 2000D] %1
\begin{enumerate}[(a)]
\item
Find a vector perpendicular at the point
$(1,1,3)$ to the surface with equation $x^2+z^2=10$.

\item 
Find a vector tangent at the same point to the curve of 
intersection of the surface in part (a) with surface $y^2+z^2=10$.

\item 
Find parametric equations for the line tangent to that curve
at that point.
\end{enumerate}
\end{question}

\begin{hint}
(b) If $\vv$ is tangent, at a point $P$, to the curve of intersection of the
surfaces $S_1$ and $S_2$, then $\vv$ 
\begin{itemize}\itemsep1pt \parskip0pt \parsep0pt %\itemindent-15pt
\item
has to be tangent to $S_1$ at $P$, and so must be perpendicular to the 
normal vector to $S_1$ at $P$ and
\item
has to be tangent to $S_2$ at $P$, and so must be perpendicular to the 
normal vector to $S_2$ at $P$.
\end{itemize}
\end{hint}

\begin{answer}
(a) $( 1,0,3)$\qquad
(b) $( 3,3,-1)$\qquad
(c) $\vr(t)=( 1,1,3)+t( 3,3,-1)$
\end{answer}

\begin{solution}
(a) 
A vector perpendicular to $x^2+z^2=10$ at $(1,1,3)$ is
\begin{equation*}
\vnabla(x^2+z^2)\big|_{(1,1,3)}
=(2x\hi+2z\hk)\big|_{(1,1,3)}
=2\hi+6\hk\hbox{ or }
\frac{1}{2} ( 2,0,6)=( 1,0,3)
\end{equation*}

(b) A vector perpendicular to $y^2+z^2=10$ at $(1,1,3)$ is
\begin{equation*}
\vnabla(y^2+z^2)\big|_{(1,1,3)}
=(2y\hj+2z\hk)\big|_{(1,1,3)}
=2\hj+6\hk\hbox{ or }\frac{1}{2} ( 0,2,6)=( 0,1,3)
\end{equation*}
A vector is tangent to the specified curve at the specified point if and only
if it  perpendicular to both $(1,0,3)$ and $(0,1,3)$. One such vector is
\begin{equation*}
( 0,1,3)\times(1,0,3)
=\det\left[\begin{matrix}
                     \hi & \hj & \hk \\
                     0   &  1  & 3 \\
                     1   &  0  & 3
                \end{matrix}\right]
=( 3,3,-1)
\end{equation*}

(c) The specified tangent line passes through $(1,1,3)$ and has direction
vector $( 1,1,3)$ and so has vector parametric equation
$$
\vr(t)=( 1,1,3)+t( 3,3,-1)
$$
\end{solution}

%%%%%%%%%%%%%%%%%%%%%%%%%%%%%%%%
\begin{question}[M200 2000A] %2
Let $P$ be the point where the curve 
\begin{equation*}
\vr(t) = t^3\,\hi + t\,\hj + t^2\,\hk,\qquad (0 \le t <\infty)
\end{equation*}
 intersects the surface 
\begin{equation*}
z^3 + xyz -2 = 0
\end{equation*}
Find the (acute) angle between the curve and the surface at $P$. 
\end{question}

\begin{hint}
 The angle between the curve and the surface at $P$ is $90^\circ$
minus the angle between the curve and the normal vector to the surface at $P$.
\end{hint}

\begin{answer}
$49.11^\circ$ (to two decimal places)
\end{answer}

\begin{solution}
$\vr(t)=( x(t)\,,\,y(t)\,,\,z(t))$ intersects $z^3 + xyz -2 = 0$ when
\begin{equation*}
z(t)^3+x(t)\,y(t)\,z(t)-2=0\iff \big(t^2\big)^3 + \big(t^3)(t)\big(t^2\big)-2=0
\iff 2t^6=2\iff t=1
\end{equation*}
since $t$ is required to be positive.
The direction vector for the curve at $t=1$ is
\begin{equation*}
\vr'(1)=3\,\hi+\hj+2\,\hk
\end{equation*}
A normal vector for the surface at $\vr(1)=( 1,1,1)$ is
\begin{equation*}
\vnabla(z^3+xyz)\big|_{(1,1,1)}=[yz\hi+xz\hj+(3z^2+xy)\hk]_{(1,1,1)}
=\hi+\hj+4\hk
\end{equation*}
The angle $\theta$ between the curve and the normal vector to the surface
is determined by
\begin{align*}
\big|( 3,1,2)\big|\,\big|( 1,1,4)\big|\cos\theta
             =( 3,1,2) \cdot( 1,1,4)
&\iff \sqrt{14}\sqrt{18}\cos\theta=12 \\
&\iff \sqrt{7\times 36}\cos\theta=12 \\
&\iff \cos\theta=\frac{2}{\sqrt{7}} \\
&\iff \theta=40.89^\circ
\end{align*}
The angle between the curve and the surface is 
$90-40.89=49.11^\circ$ (to two decimal places).
\end{solution}

%%%%%%%%%%%%%%%%%%%%%%%%%%%%%%%%
\begin{question}
Find all horizontal planes that are tangent to the surface with equation
\begin{equation*}
z=xy e^{-(x^2+y^2)/2}
\end{equation*}
What are the largest and smallest values of $z$ on this surface?
\end{question}

\begin{hint}
At the highest and lowest points of the surface, the tangent plane is horizontal.
\end{hint}

\begin{answer}
The horizontal tangent planes are $z=0$, $z=e^{-1}$ and $z=-e^{-1}$.
The largest and smallest values of $z$ are $e^{-1}$ and $-e^{-1}$, respectively.
\end{answer}

\begin{solution}
Let $(x_0,y_0,z_0)$ be any point on the surface. A vector
normal to the surface at $(x_0,y_0,z_0)$ is
\begin{align*}
&\vnabla\Big(xy e^{-(x^2+y^2)/2}-z\Big)\bigg|_{(x_0,y_0,z_0)} 
\\&\hskip1in
=\left( y_0 e^{-(x_0^2+y_0^2)/2}-x_0^2y_0 e^{-(x_0^2+y_0^2)/2},
      x_0 e^{-(x_0^2+y_0^2)/2}-x_0y_0^2 e^{-(x_0^2+y_0^2)/2},-1\right)
\end{align*}
The tangent plane to the surface at $(x_0,y_0,z_0)$ is horizontal
if and only if this vector is vertical, which is the case 
if and only if its $x$- and $y$-components are
zero, which in turn is the case if and only if
\begin{align*}
&y_0(1-x_0^2)=0\text{ and }x_0(1-y_0^2)=0\\
&\iff\big\{y_0=0\text{ or }x_0=1\text{ or }x_0=-1\big\}
   \text{ and }\big\{x_0=0\text{ or }y_0=1\text{ or }y_0=-1\big\}\\
&\iff (x_0,y_0)=(0,0)\text{ or }(1,1)\text{ or }(1,-1)
   \text{ or }(-1,1)\text{ or }(-1,-1)
\end{align*}
The values of $z_0$ at these points are $0$, $e^{-1}$, $-e^{-1}$, $-e^{-1}$ 
and $e^{-1}$, respectively. So the horizontal tangent planes are
$z=0$, $z=e^{-1}$ and $z=-e^{-1}$.
At the highest and lowest points of the surface, the tangent plane is horizontal.
So the largest and smallest values of $z$ are $e^{-1}$ and $-e^{-1}$, respectively.


\end{solution}
