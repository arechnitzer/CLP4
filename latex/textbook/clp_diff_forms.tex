
%%%%%%%%%%%%%%%%%%%%%%%%%%%%%%%%%%%%%%%%%%%%%%%%%%%%
\section{Optional --- A Generalized Stokes' Theorem}  
%\section{Optional --- A Unified Formulation of Stokes'--like Theorems}  
%\section{Optional --- A Unified Formulation of the Fundamental Theorem of Calculus, the Divergence Theorem, Greens' Theorem and Stokes' Theorem}  
     \label{sec:unified}
%%%%%%%%%%%%%%%%%%%%%%%%%%%%%%%%%%%%%%%%%%%%%%%%%%%

As we have seen, the fundamental theorem of calculus, 
the divergence theorem, Greens' theorem and Stokes' theorem
share a number of common features. There is in fact a single framework 
which encompasses and generalizes all of them, and there is a 
single theorem of which they are all special cases. We now give a 
bare bones introduction to this framework and theorem. A proper 
treatment typically takes up a good part of a full course. 
Here is an outline of what we shall do:
\begin{itemize}
\item
First, we will define ``differential forms''. To try and keep 
things as simple and concrete as possible, we'll only define\footnote{In 
general, a differential form is defined on a ``manifold'', which is 
an abstract generalization of a multi-dimensional surface, like a sphere or a torus.} 
differential forms on $\bbbr^3$ --- all of our functions will be defined on
$\bbbr^3$. Very roughly
speaking, a $k$-form is what you write after the integral sign of 
an integral over a $k$ dimensional object. Here $k$ is one of $0$, $1$, 
$2$, $3$.  As a example, a $1$-form is an expression of the
form $F_1(x,y,z)\,\dee{x}
 + F_2(x,y,z)\,\dee{y}
 + F_3(x,y,z)\,\dee{z}$. For $k=0$, think of a point as a zero 
dimensional object and think of evaluating a function at a point 
as ``integrating the function over the point''.

\item
Then we will define some operations on differential forms, so that 
we can add them, multiply them, differentiate them and, eventually,
integrate them.  The derivative of 
a $k$-form $\om$ is a $(k+1)$-form that is denoted $\dee{\om}$.
It will turn out that 
\begin{itemize}\itemsep1pt \parskip0pt \parsep0pt %\itemindent-15pt
\item[$\circ$]
differentiating a $0$-form amounts to taking a gradient,
\item[$\circ$]
differentiating a $1$-form amounts to taking a curl, and
\item[$\circ$]
differentiating a $2$-form amounts to taking a divergence.
\end{itemize}

\item
Finally we will get to the generalized Stokes' theorem which says that,
if $\om$ is a $k$-form (with $k=0,1,2$) and $D$ is a $(k+1)$-dimensional
domain of integration, then
\begin{equation*}
\int_D d\om=\int_{\partial D}\om
\end{equation*}
It will turn out that 
\begin{itemize}\itemsep1pt \parskip0pt \parsep0pt %\itemindent-15pt
\item[$\circ$]
when $k=0$, this is just the fundamental theorem of calculus and
\item[$\circ$]
when $k=1$, this is both Green's theorem and our Stokes' theorem, and
\item[$\circ$]
when $k=2$, this is the divergence theorem.
\end{itemize}
\end{itemize} 

Now let's get to work. For simplicity, we will assume throughout this 
section that all derivatives of all functions exist and are continuous.
Our first task to define differential forms. 

As we said above we will define a 1-form as an expression of the form 
$F_1(x,y,z)\,\dee{x}
 + F_2(x,y,z)\,\dee{y}
 + F_3(x,y,z)\,\dee{z}$. When you learned the definition of the 
integral the symbol ``$\dee{x}$'' was not given any mathematical 
meaning by itself. A meaning was given only to the collections of symbols 
``$\int f(x)\ \dee{x}$'' and ``$\int_a^b f(x)\ \dee{x}$''. 
Later in this section, we will give a meaning
to $\dee{x}$. We will, in Definition \ref{def:differentialFormDiff},
define a differentiation operator that we will call $\dee{}$. Then $\dee{x}$
will be that differentiation operator applied to the function $f(x)=x$.
However, until then we will have to treat $\dee{x}$ and $\dee{y}$ and
$\dee{z}$ just as symbols. Their sole role in 
$F_1(x,y,z)\,\dee{x}
 + F_2(x,y,z)\,\dee{y}
 + F_3(x,y,z)\,\dee{z}$
is to allow us to distinguish\footnote{We could also define, 
for example, a $1$-form as an ordered list 
   $\big( F_1(x,y,z)\,,\,
          F_2(x,y,z)\,,\,
          F_3(x,y,z)\big)$ of three functions and just view
$ F_1(x,y,z)\,\dee{x}
 + F_2(x,y,z)\,\dee{y}
 + F_3(x,y,z)\,\dee{z}$
as another notation for 
    $\big( F_1(x,y,z)\,,\,
          F_2(x,y,z)\,,\,
          F_3(x,y,z)\big)$.} 
$F_1(x,y,z)$, $F_2(x,y,z)$ and $F_3(x,y,z)$.

Similarly, we will define a 2-form as an expression of the form 
$F_1(x,y,z)\,\dee{y}\wedge\dee{z}
 + F_2(x,y,z)\,\dee{z}\wedge\dee{x}
 + F_3(x,y,z)\,\dee{x}\wedge\dee{y}$. 
Once again there is a symbol, namely ``$\wedge$'', that we have not yet given a meaning to. We will, in Definition \ref{def:differentialFormMult},
define a product, called the wedge product, with $\wedge$ as 
the multiplication symbol. 
Then $\dee{x}\wedge\dee{y}$ will be the wedge product of $\dee{x}$ and 
$\dee{y}$. Until then we will have to treat $\dee{y}\wedge\dee{z}$,
$\dee{z}\wedge\dee{x}$ and $\dee{x}\wedge\dee{y}$ just as three more
meaningless symbols.

Finally here is the definition.

\begin{defn}\label{def:differentialForm}
\begin{enumerate}[(a)]
\item 
A $0$-form is a function $f(x,y,z)$.

\item
A $1$-form is an expression of the form
\begin{equation*} 
  F_1(x,y,z)\,\dee{x}
 + F_2(x,y,z)\,\dee{y}
 + F_3(x,y,z)\,\dee{z}
\end{equation*}
with $F_1(x,y,z)$, $F_2(x,y,z)$ and $F_3(x,y,z)$ being functions
of three variables.

\item
A $2$-form is an expression of the form 
\begin{equation*}
 F_1(x,y,z)\,\dee{y}\wedge\dee{z}
 + F_2(x,y,z)\,\dee{z}\wedge\dee{x}
 + F_3(x,y,z)\,\dee{x}\wedge\dee{y}
\end{equation*}
with $F_1(x,y,z)$, $F_2(x,y,z)$ and $F_3(x,y,z)$ being functions
of three variables.

\item
A $3$-form is an expression of the form 
 $ f(x,y,z)\,\dee{x}\wedge\dee{y}\wedge\dee{z}$,
with $f(x,y,z)$ being a function of three variables.

\end{enumerate}
At this stage (there'll more later), just think of ``$\dee{x}$'', 
``$\dee{y}$'', ``$\dee{z}$'', ``$\dee{x}\wedge\dee{y$}'', and so on, 
as symbols.  Do not yet attempt to attach any significance to them. 
\end{defn}

There are four operations involving differential forms --- addition,
multiplication ($\wedge$), differentiation ($\dee{}$) and integration. 
Here are their
definitions. First, addition is defined, and works, just the way that 
you would expect it to.

\begin{defn}[Addition of differential forms]\label{def:differentialFormAdd}
\begin{enumerate}[(a)]
\item 
The sum of the $0$-forms  $f$ and $g$ is the $0$-form $f+g$.

\item
The sum of two $1$-forms is the $1$-form
\begin{align*} 
  &\big[F_1\,\dee{x}
          + F_2\,\dee{y}
          + F_3\,\dee{z}\big] \\
  +&\big[G_1\,\dee{x}
          + G_2\,\dee{y}
          + G_3\,\dee{z}\big] \\
 =& (F_1+G_1)\,\dee{x}
          + (F_2+G_2)\,\dee{y}
          + (F_3+G_3)\,\dee{z} 
\end{align*}

\item
The sum of two $2$-forms is the $2$-form
\begin{align*} 
  &\big[F_1\,\dee{y}\wedge\dee{z}
          + F_2\,\dee{z}\wedge\dee{x}
          + F_3\,\dee{x}\wedge\dee{y}\big] \\
  +&\big[G_1\,\dee{y}\wedge\dee{z}
          + G_2\,\dee{z}\wedge\dee{x}
          + G_3\,\dee{x}\wedge\dee{y}\big] \\
  =&(F_1+G_1)\,\dee{y}\wedge\dee{z}
          + (F_2+G_2)\,\dee{z}\wedge\dee{x}
          + (F_3+G_3)\,\dee{x}\wedge\dee{y} 
\end{align*}

\item
The sum of two $3$-forms is the $3$-form
\begin{align*} 
   f\,\dee{x}\wedge\dee{y}\wedge\dee{z} 
 \ +\  g\,\dee{x}\wedge\dee{y}\wedge\dee{z} 
 \ =\  \big(f+g\big)\,\dee{x}\wedge\dee{y}\wedge\dee{z}
\end{align*}
\end{enumerate}
\end{defn}


There is one wrinkle in multiplication. It is not commutative,
meaning that $\alpha\wedge\be$ need not be the same as $\be\wedge\alpha$.
You have already seen some noncommutative products. If $\va$ and $\vb$
are two vectors in $\bbbr^3$, then $\va\times\vb = -\vb\times \va$.
Also, if $A$ and $B$ are two $n\times n$ matrices, the matrix product 
$AB$ need not be the same  as $BA$. 

\begin{defn}[Multiplication of differential forms]
             \label{def:differentialFormMult}
We now define a multiplication rule for differential forms. 
If $\om$ is a $k$-form and $\om'$ is a 
$k'$-form then the product will be a $(k+k')$-form and will be denoted
$\om\wedge\om'$ (read ``omega wedge omega prime''). It is 
determined by the following properties.
\begin{enumerate}[(a)]
\item
If $f$ is a function (i.e. a $0$-form), then
\begin{align*}
 f\big[F_1\,\dee{x} + F_2\,\dee{y} + F_3\,\dee{z}\big]
   &= (fF_1)\,\dee{x} + (fF_2)\,\dee{y} + (fF_3)\,\dee{z}
\\
 f\big[F_1\,\dee{y}\wedge\dee{z}
       + F_2\,\dee{z}\wedge\dee{x} 
       + F_3\,\dee{x}\wedge\dee{y}\big]
  &=(fF_1)\,\dee{y}\wedge\dee{z}
       + (fF_2)\,\dee{z}\wedge\dee{x} \\&\hskip1.25in
       + (fF_3)\,\dee{x}\wedge\dee{y}
\\
 f\big[g\,\dee{x}\wedge\dee{y}\wedge\dee{z}\big]
   &= (fg)\,\dee{x}\wedge\dee{y}\wedge\dee{z}
\end{align*}
Traditionally, the $\wedge$ is not written when multiplying
a differential form by a function (i.e. a $0$-form).

\item[(b)] % (b)
$\om\wedge\om'$ is linear in $\om$ and in $\om'$. This means that
if $\om = f_1\om_1+f_2\om_2$, where $f_1$,$f_2$ are 
functions and $\om_1,\om_2$ are forms, then
\begin{equation*}
\big(f_1\om_1+f_2\om_2\big)\wedge \om'= f_1 (\om_1\wedge\om')
+f_2 (\om_2\wedge\om')
\end{equation*}
Similarly, 
\begin{equation*}
\om\wedge\big(f_1\om'_1+f_2\om'_2\big)= f_1 (\om\wedge\om'_1)
+f_2 (\om\wedge\om'_2)
\end{equation*}


\item[(c)] %% (c)  
%The product is graded anticommutative. This means that 
If $\om$ is a $k$-form and $\om'$ is a  $k'$-form then \begin{equation*}
   \om\wedge\om'=(-1)^{kk'}\om'\wedge\om
\end{equation*}
That is, if at least one of $k$ and $k'$ is even, then
\begin{equation*}
\om\wedge\om'=\om'\wedge\om
\end{equation*}
(so that the wedge product is commutative) 
and if $k$ and $k'$ are \emph{both odd} then 
\begin{equation*}
\om\wedge\om'=-\om'\wedge\om
\end{equation*} 
(so that the wedge product is anticommutative).
In particular, if $\om$ is a $d$-form with $d$ \emph{odd}
\begin{equation*}
\om\wedge\om = 0
\end{equation*}

%\item[(d)] %% (d) 
%The wedge product is associative. This means that
%\begin{equation*}
%(\om\wedge\om')\wedge\om''=\om\wedge\big(\om'\wedge\om''\big)
%\end{equation*}



\end{enumerate}
\end{defn}

\addtocounter{theorem}{-1}
\begin{defn}[continued]
\begin{enumerate}[(a)]

\item[(d)] %% (d) 
The wedge product is associative. This means that
\begin{equation*}
(\om\wedge\om')\wedge\om''=\om\wedge\big(\om'\wedge\om''\big)
\end{equation*}

\end{enumerate}
\end{defn}


%\begin{defn}[Multiplication of differential forms]
%             \label{def:differentialFormMultB}
%The product of the differential forms $\alpha$ and $\be$ is written
%$\alpha\wedge\be$. If $\al$ is a $d$-form and $\be$ is a $d'$-form,
%then $\al\wedge\be$ is a $(d+d')$-form. This product obeys most of 
%the usual multiplication rules, like
%\begin{equation*}
%(\alpha+\be)\wedge\ga = \alpha\wedge\ga + \be\wedge\ga\qquad
%\big(\al\wedge\be\big)\wedge\ga = \alpha\wedge\big(\be\wedge\ga\big)
%\end{equation*}
%However there are two exceptions:
%\begin{enumerate}[(a)]
%\item 
%If either $\alpha$ or $\be$ are $0$-forms, the multiplication symbol
%$\wedge$ is usually not written. For example, the product of the $0$-form
%$f(x,y,z)$ and the $1$-form $\dee{x}$ is usually written $f(x,y,z)\,\dee{x}$.
%
%\item
%If $\alpha$ is a $d$-form and $\be$ are $d'$-form with $d$ and $d'$
%\emph{both odd} then
%\begin{equation*}
%\alpha\wedge\be = -\be\wedge\alpha
%\end{equation*}
%In particular, if $\alpha$ is a $d$-form with $d$ \emph{odd}
%\begin{equation*}
%\al\wedge\al = 0
%\end{equation*}
%
%\end{enumerate}
%\end{defn}

\noindent
So the wedge product obeys most of the usual multiplication rules, with
the one big exception that if $\om$ is $k$-form and $\om'$ is a 
$k'$-form with $k$ and $k'$  \emph{both odd} then $\om\wedge\om'=-\om'\wedge\om$. 

The best way to get a handle on
the wedge product is to work through some examples, like these.

%\pagebreak[2]
\goodbreak
\begin{eg}\label{eg:diffFormAddMultA}
Let 
    $\om = F_1\,\dee{x}
     + F_2\,\dee{y}
     + F_3\,\dee{z}
    $
and
    $\om' = G_1\,\dee{x}
     + G_2\,\dee{y}
     + G_3\,\dee{z}
    $
be any two $1$-forms. Their product is
\begin{align*}
   \om\wedge\om'
&=\big[F_1\,\dee{x}
     + F_2\,\dee{y}
     + F_3\,\dee{z}\big]
   \wedge
   \big[G_1\,\dee{x}
     + G_2\,\dee{y}
     + G_3\,\dee{z}\big] \\
   &= \ \big(F_1\,\dee{x}\big)\wedge\big(G_1\,\dee{x}\big)
     +\big(F_1\,\dee{x}\big)\wedge\big(G_2\,\dee{y}\big)
     +\big(F_1\,\dee{x}\big)\wedge\big(G_3\,\dee{z}\big) \\
   &\hskip0.1in +\big(F_2\,\dee{y}\big)\wedge\big(G_1\,\dee{x}\big)
     +\big(F_2\,\dee{y}\big)\wedge\big(G_2\,\dee{y}\big)
     +\big(F_2\,\dee{y}\big)\wedge\big(G_3\,\dee{z}\big) \\
   &\hskip0.1in+ \big(F_3\,\dee{z}\big)\wedge\big(G_1\,\dee{x}\big)
     +\big(F_3\,\dee{z}\big)\wedge\big(G_2\,\dee{y}\big)
     +\big(F_3\,\dee{z}\big)\wedge\big(G_3\,\dee{z}\big) \\
   &\hskip0.5in\text{(by linearity, i.e. by part (b) of 
                      Definition \ref{def:differentialFormMult})} 
\displaybreak[0]\\
   &= \ F_1G_1\,\dee{x}\wedge\,\dee{x}
     + F_1G_2\,\dee{x}\wedge\,\dee{y}
     + F_1G_3\,\dee{x}\wedge\,\dee{z} \\
   &\hskip0.1in +F_2G_1\,\dee{y}\wedge\,\dee{x}
     + F_2G_2\,\dee{y}\wedge\,\dee{y}
     + F_2G_3\,\dee{y}\wedge\,\dee{z} \\
   &\hskip0.1in+ F_3G_1\,\dee{z}\wedge\,\dee{x}
     + F_3G_2\,\dee{z}\wedge\,\dee{y}
     + F_3G_3\,\dee{z}\wedge\,\dee{z} \\
   &= \big(F_1G_2-F_2G_1)\,\dee{x}\wedge\dee{y}
                  +\big(F_3G_1-F_1G_3)\,\dee{z}\wedge\dee{x}
                  +\big(F_2G_3-F_3G_2)\,\dee{y}\wedge\dee{z}
\end{align*}
because 
\begin{equation*}
\dee{x}\wedge\,\dee{x}=\dee{y}\wedge\,\dee{y}=\dee{z}\wedge\,\dee{z}=0
\end{equation*}
and
\begin{equation*}
\dee{x}\wedge\,\dee{y}=-\dee{y}\wedge\,\dee{x}\qquad
\dee{x}\wedge\,\dee{z}=-\dee{z}\wedge\,\dee{x}\qquad
\dee{z}\wedge\,\dee{y}=-\dee{y}\wedge\,\dee{z}
\end{equation*}
Note that, if we view $\vF=(F_1,F_2,F_3)$ and $\vG=(G_1,G_2,G_3)$ 
as vectors, we can write the product simply as 
\begin{impeqn}\label{eqn:prodCross}
\begin{align*}
&\big[F_1\,\dee{x}
     + F_2\,\dee{y}
     + F_3\,\dee{z}\big]
   \wedge
   \big[G_1\,\dee{x}
     + G_2\,\dee{y}
     + G_3\,\dee{z}\big]
\\
&\hskip1in=(\vF\times\vG)_1\, \dee{y}\wedge\dee{z}
 +(\vF\times\vG)_2\, \dee{z}\wedge\dee{x}
 +(\vF\times\vG)_3\, \dee{x}\wedge\dee{y}
\end{align*}
\end{impeqn}
\noindent
where we are using $(\vF\times\vG)_\ell$ to denote the $\ell^{\rm th}$
component of the cross product $\vF\times\vG$.
In the special case that $F_3=G_3=0$, we have
\begin{impeqn}\label{eqnprodDet}
\begin{align*}
\big[F_1\,\dee{x}
     + F_2\,\dee{y}\big]
   \wedge
   \big[G_1\,\dee{x}
     + G_2\,\dee{y}\big]
=\big(F_1G_2-F_2G_1)\,\dee{x}\wedge\dee{y}
=\det\left[\begin{matrix} F_1 & F_2\\ G_1 & G_2\end{matrix}\right]
         \dee{x}\wedge\dee{y}
\end{align*}
\end{impeqn}

We can now see why in the Definition \ref{def:differentialForm}.c
of $2$-forms
\begin{itemize}\itemsep1pt \parskip0pt \parsep0pt %\itemindent-15pt
\item[$\circ$]
there were no $\dee{x}\wedge\dee{x}$ or
              $\dee{y}\wedge\dee{y}$ or
              $\dee{z}\wedge\dee{z}$ 
terms --- they are all zero and
\item[$\circ$]
there were no $\dee{y}\wedge\dee{x}$ or
              $\dee{z}\wedge\dee{y}$ or
              $\dee{x}\wedge\dee{z}$ 
terms --- they can all be rewritten using 
$\dee{x}\wedge\dee{y}$,
              $\dee{y}\wedge\dee{z}$ and
              $\dee{z}\wedge\dee{x}$ 
terms (or vice versa).
\end{itemize}
The reason that we chose to write the Definition \ref{def:differentialForm}.c
as\begin{equation*}
 F_1\,\dee{y}\wedge\dee{z}
 + F_2\,\dee{z}\wedge\dee{x}
 + F_3\,\dee{x}\wedge\dee{y}
\end{equation*}
as opposed to in the form, for example,
\begin{equation*}
 f_1\,\dee{x}\wedge\dee{y}
 + f_2\,\dee{x}\wedge\dee{z}
 + f_3\,\dee{y}\wedge\dee{z}
\end{equation*}
was to make formulae like \eqref{eqn:prodCross} work. 
The easy way to remember
\begin{equation*}
 F_1\,\dee{y}\wedge\dee{z}
 + F_2\,\dee{z}\wedge\dee{x}
 + F_3\,\dee{x}\wedge\dee{y}
\end{equation*}
is to rename (in your head) $x,y,z$ to $x_1,x_2,x_3$. Then the subscripts
in the three terms of 
\begin{equation*}
 F_1\,\dee{x_2}\wedge\dee{x_3}
 + F_2\,\dee{x_3}\wedge\dee{x_1}
 + F_3\,\dee{x_1}\wedge\dee{x_2}
\end{equation*}
are just $1,2,3$ and $2,3,1$ and $3,1,2$ --- the three cyclic permutations
of $1,2,3$.
\end{eg}

\begin{eg}\label{eg:diffFormAddMultB}
The product of the (general) $1$-form
    $\om = F_1\,\dee{x}
     + F_2\,\dee{y}
     + F_3\,\dee{z}
    $
and the (general) $2$-form
    $\om'=\big[G_1\,\dee{y}\wedge\dee{z}
          + G_2\,\dee{z}\wedge\dee{x}
          + G_3\,\dee{x}\wedge\dee{y}\big]
    $
(again note the numbering of the coefficients in the $2$-form) is
\begin{align*}
  \om\wedge\om'
 &=\big[F_1\,\dee{x}
     + F_2\,\dee{y}
     + F_3\,\dee{z}\big]
   \wedge
   \big[G_1\,\dee{y}\wedge\dee{z}
          + G_2\,\dee{z}\wedge\dee{x}
          + G_3\,\dee{x}\wedge\dee{y}\big] \\
  &= \ F_1G_1\,\dee{x}\wedge\dee{y}\wedge\dee{z}
     + F_2G_2\,\dee{y}\wedge\dee{z}\wedge\dee{x}
     + F_3G_3\,\dee{z}\wedge\dee{x}\wedge\dee{y} \\
  & = \big(F_1G_1+F_2G_2+F_3G_3)\,\dee{x}\wedge\dee{y}\wedge\dee{z}
%\\
%  &\hskip0.5in = \vF\cdot\vG\,\dee{x}\wedge\dee{y}\wedge\dee{z}
\end{align*}
Here we have used that, for $1$-forms, $\alpha\wedge\beta=-\beta\wedge\alpha$,
so that 
\begin{align*}
\dee{y}\wedge\dee{z}\wedge\dee{x}
&=-\dee{y}\wedge\dee{x}\wedge\dee{z}=\dee{x}\wedge\dee{y}\wedge\dee{z}
\\
\dee{z}\wedge\dee{x}\wedge\dee{y} 
&=-\dee{x}\wedge\dee{z}\wedge\dee{y} =\dee{x}\wedge\dee{y}\wedge\dee{z} 
\end{align*}
We have also used that any wedge product of three 
$\dee{\{x\text{ or }y\text{ or }z\}}$'s
with at least two of the coordinates being the same is zero. For example
\begin{equation*}
\dee{x}\wedge\dee{z}\wedge\dee{x} = - \dee{x}\wedge\dee{x}\wedge\dee{z}
   =0
\end{equation*}
So 
\begin{impeqn}\label{eqnProd1Form2Form}
\begin{align*}
   &\big[F_1\,\dee{x}
     + F_2\,\dee{y}
     + F_3\,\dee{z}\big]
   \wedge
   \big[G_1\,\dee{y}\wedge\dee{z}
          + G_2\,\dee{z}\wedge\dee{x}
          + G_3\,\dee{x}\wedge\dee{y}\big] \\
  &\hskip4in = \vF\cdot\vG\,\dee{x}\wedge\dee{y}\wedge\dee{z}
\end{align*}
\end{impeqn}
\end{eg}

\begin{eg}\label{eg:diffFormAddMultC}
Combining Examples \ref{eg:diffFormAddMultA} and \ref{eg:diffFormAddMultB}, 
we have the wedge product of any three (general)
$1$-forms
    $F_1\,\dee{x}
     + F_2\,\dee{y}
     + F_3\,\dee{z}$
and
    $G_1\,\dee{x}
     + G_2\,\dee{y}
     + G_3\,\dee{z}$
and 
    $H_1\,\dee{x}
     + H_2\,\dee{y}
     + H_3\,\dee{z}$
is
\begin{align*}
&\big[F_1\,\dee{x}
     + F_2\,\dee{y}
     + F_3\,\dee{z}\big]
   \wedge
   \big[G_1\,\dee{x}
     + G_2\,\dee{y}
     + G_3\,\dee{z}\big]
   \wedge
   \big[H_1\,\dee{x}
     + H_2\,\dee{y}
     + H_3\,\dee{z}\big]
\\
&\hskip0.1in=\big[F_1\,\dee{x}
     + F_2\,\dee{y}
     + F_3\,\dee{z}\big]
   \wedge
   \big[(\vG\times\vH)_1\, \dee{y}\wedge\dee{z}
 +(\vG\times\vH)_2\, \dee{z}\wedge\dee{x}
 +(\vG\times\vH)_3\, \dee{x}\wedge\dee{y}\big]
\\
&\hskip0.1in
  =\big\{F_1(\vG\times\vH)_1 + F_2(\vG\times\vH)_2 + F_3(\vG\times\vH)_3\big\}
     \dee{x}\wedge\dee{y}\wedge\dee{z}
\\
&\hskip0.1in
  =\big\{F_1(G_2H_3-G_3H_2) + 
         F_2(G_3H_1-G_1H_3) + 
         F_3(G_1H_2-G_2H_1\big\}
     \dee{x}\wedge\dee{y}\wedge\dee{z}
%\\
%&\hskip0.1in
%  =\det\left[\begin{matrix} F_1 & F_2 & F_3 \\ 
%                            G_1 & G_2 & G_3 \\
%                            H_1 & H_2 & H_3 \end{matrix}\right]
%     \dee{x}\wedge\dee{y}\wedge\dee{z}
\end{align*}
This can be expressed cleanly in terms of determinants.
Recalling the rule for expanding a determinant along its top row
\begin{impeqn}\label{eqnProd3Form}
\begin{align*}
&\big[F_1\,\dee{x}
     + F_2\,\dee{y}
     + F_3\,\dee{z}\big]
   \wedge
   \big[G_1\,\dee{x}
     + G_2\,\dee{y}
     + G_3\,\dee{z}\big]
   \wedge
   \big[H_1\,\dee{x}
     + H_2\,\dee{y}
     + H_3\,\dee{z}\big]
\\
&\hskip3.5in
  =\det\left[\begin{matrix} F_1 & F_2 & F_3 \\ 
                            G_1 & G_2 & G_3 \\
                            H_1 & H_2 & H_3 \end{matrix}\right]
     \dee{x}\wedge\dee{y}\wedge\dee{z}
\end{align*}
\end{impeqn}
\end{eg}

Our next operation is a differential operator which unifies and generalizes
gradient, curl and divergence.

\begin{defn}[Differentiation of differential forms]
             \label{def:differentialFormDiff}
If $\om$ is a $k$-form, then $\dee{\om}$ is a $k+1$-form, with
$\dee{}$ being the unique\footnote{That $\dee{}$ is unique just means that
the action of $\dee{}$ on \emph{any} differential form is completely 
determined by the four rules (a), (b), (c), (d). We will see in 
Example \ref{eg:diffFormDiff}.b,c,d, that this is indeed the case.} 
such operator that obeys
\begin{enumerate}[(a)]
\item
$\dee{}$ is linear. That is, if $\om_1,\om_2$ are $k$-forms and 
$a_1,a_2\in\bbbr$, then 
\begin{equation*}
\dee{\big(a_1\om_1+a_2\om_2\big)}
=a_1\dee{\om_1}+a_2\dee{\om_2}
\end{equation*}

\item
$\dee{}$ obeys a ``graded product rule''. Precisely, if $\om^{(k)}$ is a 
$k$-form and $\om^{(\ell)}$ is an $\ell$-form, then 
\begin{equation*}
\dee{\big(\om^{(k)}\wedge\om^{(\ell)}\big)}
=\big(d\om^{(k)}\big)\wedge\om^{(\ell)}
+(-1)^k\om^{(k)} \wedge \big(\dee{\om^{(\ell)}}\big)
\end{equation*}

\item
If $f(x,y,z)$ is a $0$-form, then
\begin{align*}
\dee{f}
&=\frac{\partial f}{\partial x}(x,y,z)\ \dee{x}
+\frac{\partial f}{\partial y}(x,y,z)\ \dee{y}
+\frac{\partial f}{\partial z}(x,y,z)\ \dee{z} \\
&=\vnabla f(x,y,z)\cdot\dee{\vr}
\qquad\text{where }
\dee{\vr} = \dee{x}\,\hi + \dee{y}\,\hj + \dee{z}\,\hk
\end{align*}

\item
For any differential form $\om$, 
\begin{equation*}
\dee{\big(\dee{\om}\big)}=0
\end{equation*}
\end{enumerate}
\end{defn}

\begin{eg}\label{eg:diffFormDiff}
\begin{enumerate}[(a)]

\item %(a)
If $f(x,y,z) = x$, then
\begin{equation*}
\dee{f}
=\frac{\partial x}{\partial x}(x,y,z)\ \dee{x}
+\frac{\partial x}{\partial y}(x,y,z)\ \dee{y}
+\frac{\partial x}{\partial z}(x,y,z)\ \dee{z} 
=\dee{x}
\end{equation*}
That is, $\dee{x}$ really is the operator $\dee{}$ applied to the 
function $x$. Similarly, $\dee{y}$ really is the operator $\dee{}$ 
applied to the function $y$ and $\dee{z}$ really is the operator 
$\dee{}$ applied to the function $z$.

\item  %(b)
For any $k$-form $\om$
\begin{align*}
\dee{\big[\om\wedge\dee{x}\big]}
&=\dee{\om}\wedge\dee{x} + (-1)^k\om\wedge\dee{\big(\dee{x}\big)} \\
&=\dee{\om}\wedge\dee{x}
\end{align*}
Similarly
\begin{equation*}
\dee{\big[\om\wedge\dee{y}\big]}=\dee{\om}\wedge\dee{y}\qquad
\dee{\big[\om\wedge\dee{z}\big]}=\dee{\om}\wedge\dee{z}
\end{equation*}

\item %(c)
For any $1$-form
\begin{align*}
&\dee{}\big[F_1\dee{x} + F_2\dee{y} + F_3\dee{z}\big]
=\dee{F_1}\wedge\dee{x} + \dee{F_2}\wedge\dee{y}  + \dee{F_3}\wedge\dee{z}
\\
&\hskip0.5in=\Big(\frac{\partial F_1}{\partial x}\ \dee{x}
      +\frac{\partial F_1}{\partial y}\ \dee{y}
      +\frac{\partial F_1}{\partial z}\ \dee{z}\Big)\wedge\dee{x}
+\Big(\frac{\partial F_2}{\partial x}\ \dee{x}
      +\frac{\partial F_2}{\partial y}\ \dee{y}
      +\frac{\partial F_2}{\partial z}\ \dee{z}\Big)\wedge\dee{y}
   \\&\hskip2in
+\Big(\frac{\partial F_3}{\partial x}\ \dee{x}
      +\frac{\partial F_3}{\partial y}\ \dee{y}
      +\frac{\partial F_3}{\partial z}\ \dee{z}\Big)\wedge\dee{z}
\\
&\hskip0.5in=
  \Big(\frac{\partial F_3}{\partial y}-\frac{\partial F_2}{\partial z}\Big)
          \ \dee{y}\wedge\dee{z}
   +\Big(\frac{\partial F_1}{\partial z}-\frac{\partial F_3}{\partial x}\Big)
          \ \dee{z}\wedge\dee{x}
   +\Big(\frac{\partial F_2}{\partial x}-\frac{\partial F_1}{\partial y}\Big)
          \ \dee{x}\wedge\dee{y}
\\
&\hskip0.5in=
   (\vnabla\times\vF)_1\,\dee{y}\wedge\dee{z}
   +(\vnabla\times\vF)_2\,\dee{z}\wedge\dee{x}
   +(\vnabla\times\vF)_3\,\dee{x}\wedge\dee{y}
\end{align*}

\item %(d)
For any $2$-form
\begin{align*}
&\dee{}\big[F_1\,\dee{y}\wedge\dee{z}
 + F_2\,\dee{z}\wedge\dee{x}
 + F_3\,\dee{x}\wedge\dee{y}\big]
\\
&\hskip0.5in=\dee{F_1}\wedge\dee{y}\wedge\dee{z} 
+ \dee{F_2}\wedge\dee{z}\wedge\dee{x}  
+ \dee{F_3}\wedge\dee{x}\wedge\dee{y}\big]
\\
&\hskip0.5in=\Big(\frac{\partial F_1}{\partial x}\ \dee{x}
      +\frac{\partial F_1}{\partial y}\ \dee{y}
      +\frac{\partial F_1}{\partial z}\ \dee{z}\Big)\wedge\dee{y}\wedge\dee{z}
   \\&\hskip2in
+\Big(\frac{\partial F_2}{\partial x}\ \dee{x}
      +\frac{\partial F_2}{\partial y}\ \dee{y}
      +\frac{\partial F_2}{\partial z}\ \dee{z}\Big)\wedge\dee{z}\wedge\dee{x}
   \\&\hskip2in
+\Big(\frac{\partial F_3}{\partial x}\ \dee{x}
      +\frac{\partial F_3}{\partial y}\ \dee{y}
      +\frac{\partial F_3}{\partial z}\ \dee{z}\Big)\wedge\dee{x}\wedge\dee{y}
\\
&\hskip0.5in=
  \Big(\frac{\partial F_1}{\partial x}
       +\frac{\partial F_2}{\partial y}
       +\frac{\partial F_3}{\partial z}\Big)
          \ \dee{x}\wedge\dee{y}\wedge\dee{z}
\\
&\hskip0.5in=
   \vnabla\cdot\vF\ \dee{x}\wedge\dee{y}\wedge\dee{z}
\end{align*}

\item %(e)
For any $3$-form
\begin{align*}
\dee{}\big[f\,\dee{x}\wedge\dee{y}\wedge\dee{z}\big]
&=\Big(\frac{\partial f}{\partial x}\ \dee{x}
      +\frac{\partial f}{\partial y}\ \dee{y}
      +\frac{\partial f}{\partial z}\ \dee{z}\Big)
    \wedge\dee{x}\wedge\dee{y}\wedge\dee{z}\\
&=0
\end{align*}
\end{enumerate}
\end{eg}

\begin{eg}\label{eg:diffFormDiffB}
In Definition \ref{def:differentialFormDiff}.c, we defined,
for any function $f(x,y,z)$ of three variables
\begin{align*}
\dee{f}
&=\frac{\partial f}{\partial x}(x,y,z)\ \dee{x}
+\frac{\partial f}{\partial y}(x,y,z)\ \dee{y}
+\frac{\partial f}{\partial z}(x,y,z)\ \dee{z}
\end{align*}
The analogous formulae\footnote{Indeed, you can view $f(t)$ as a function 
of three variables that happens to be independent of two of the three 
variables. Similarly you can view $f(u,v)$  as a function of three 
variables that happens to be independent of one of the three variables.
} for functions of one or two variables also apply.
\begin{align*}
\dee{f(t)} & = \diff{f}{t}(t)\,\dee{t} \\
\dee{f(uv)}
&=\frac{\partial f}{\partial u}(u,v)\ \dee{u}
+\frac{\partial f}{\partial v}(u,v)\ \dee{v}
\end{align*}

\begin{enumerate}[(a)]
\item 
Let $F_1(x,y,z)\,\dee{x} + F_2(x,y,z)\,\dee{y} + F_3(x,y,z)\,\dee{z}$ 
be a $1$-form. Suppose that we substitute $x=x(t)$, $y=y(t)$ and $z=z(t)$,
so that we are restricting our $1$-form to a parametrized curve.
Then, writing $\vr(t) = \big(x(t),y(t),z(t)\big)$,
\begin{align*}
&F_1\big(x(t),y(t),z(t)\big)\,\dee{x(t)} 
  + F_2\big(x(t),y(t),z(t)\big)\,\dee{y(t)} 
  + F_3\big(x(t),y(t),z(t)\big)\,\dee{z(t)} \\
&\hskip0.5in=F_1\big(\vr(t)\big)\diff{x}{t}(t)\,\dee{t}
  + F_2\big(\vr(t)\big)\diff{y}{t}(t)\,\dee{t}
  + F_3\big(\vr(t)\big)\diff{z}{t}(t)\,\dee{t} \\
&\hskip0.5in= \vF\big(\vr(t)\big)\cdot\diff{\vr}{t}(t)\,\dee{t}
\end{align*}


\item 
Let $F_1(x,y,z)\,\dee{y}\wedge\dee{z}
 + F_2(x,y,z)\,\dee{z}\wedge\dee{x}
 + F_3(x,y,z)\,\dee{x}\wedge\dee{y}$ 
be a $2$-form. Suppose that we substitute $x=x(u,v)$, $y=y(u,v)$ and 
$z=z(u,v)$, so that we are restricting our $2$-form to a parametrized surface.
Then, writing $\vr(u,v) = \big(x(u,v),y(u,v),z(u,v)\big)$,
\begin{align*}
&F_1\big(x(u,v),y(u,v),z(u,v)\big)\,\dee{y(u,v)}\wedge\dee{z(u,v)}
   \\&\hskip1in
  + F_2\big(x(u,v),y(u,v),z(u,v)\big)\,\dee{z(u,v)}\wedge\dee{x(u,v)} 
   \\&\hskip1in
  + F_3\big(x(u,v),y(u,v),z(u,v)\big)\,\dee{x(u,v)}\wedge\dee{y(u,v)} \\
&=F_1\big(\vr(u,v)\big)\,
    \Big(\frac{\partial y}{\partial u}\dee{u}
        +\frac{\partial y}{\partial v}\dee{v}\Big)\wedge
    \Big(\frac{\partial z}{\partial u}\dee{u}
        +\frac{\partial z}{\partial v}\dee{v}\Big)
   \\&\hskip1in
  + F_2\big(\vr(u,v)\big)\,
    \Big(\frac{\partial z}{\partial u}\dee{u}
        +\frac{\partial z}{\partial v}\dee{v}\Big)\wedge
    \Big(\frac{\partial x}{\partial u}\dee{u}
        +\frac{\partial x}{\partial v}\dee{v}\Big) 
   \\&\hskip1in
  + F_3\big(\vr(u,v)\big)\,
    \Big(\frac{\partial x}{\partial u}\dee{u}
        +\frac{\partial x}{\partial v}\dee{v}\Big)\wedge
    \Big(\frac{\partial y}{\partial u}\dee{u}
        +\frac{\partial y}{\partial v}\dee{v}\Big) \\
&=\Big[F_1\big(\vr(u,v)\big)\,
    \Big(\frac{\partial y}{\partial u}\frac{\partial z}{\partial v}
        -\frac{\partial y}{\partial v}\frac{\partial z}{\partial u}\Big)
      +F_2\big(\vr(u,v)\big)\,
    \Big(\frac{\partial z}{\partial u}\frac{\partial x}{\partial v}
        -\frac{\partial z}{\partial v}\frac{\partial x}{\partial u}\Big)
   \\&\hskip1in
      +F_3\big(\vr(u,v)\big)\,
    \Big(\frac{\partial x}{\partial u}\frac{\partial y}{\partial v}
        -\frac{\partial x}{\partial v}\frac{\partial y}{\partial u}\Big)
        \Big] \dee{u}\wedge\dee{v} \\
&=\Big[\vF\big(\vr(u,v)\big)\cdot \frac{\partial\vr}{\partial u}(u,v)
                 \times \frac{\partial\vr}{\partial v}(u,v)\Big] 
         \dee{u}\wedge\dee{v}
\end{align*}


\end{enumerate}
\end{eg}


Let us summarize what we have seen in the Example \ref{eg:diffFormDiff}.
\begin{lemma}\label{lemma:dGradCurlDiv}
\begin{enumerate}[(a)]

\item
For any $0$-form
\begin{equation*}
\dee{f}
=\vnabla f(x,y,z)\cdot\dee{\vr}
\end{equation*}

\item
For any $1$-form
\begin{align*}
&\dee{}\big[F_1\dee{x} + F_2\dee{y} + F_3\dee{z}\big]
\\&\hskip1in
=   (\vnabla\times\vF)_1\,\dee{y}\wedge\dee{z}
   +(\vnabla\times\vF)_2\,\dee{z}\wedge\dee{x}
   +(\vnabla\times\vF)_3\,\dee{x}\wedge\dee{y}
\end{align*}

\item
For any $2$-form
\begin{equation*}
\dee{}\big[F_1\,\dee{y}\wedge\dee{z}
 + F_2\,\dee{z}\wedge\dee{x}
 + F_3\,\dee{x}\wedge\dee{y}\big]
=   \vnabla\cdot\vF\ \dee{x}\wedge\dee{y}\wedge\dee{z}
\end{equation*}

\item
For any $3$-form
\begin{equation*}
\dee{}\big[f\,\dee{x}\wedge\dee{y}\wedge\dee{z}\big]=0
\end{equation*}

\end{enumerate}
\end{lemma}

Our final operation is integration of differential forms.

\begin{defn}[Integration of differential forms]
             \label{def:differentialFormInt}
\begin{enumerate}[(a)]

\item
Let $f(x,y,z)$ be a  $0$-form and $P=(x_0,y_0,z_0)\in\bbbr^3$ be a point.
Then
\begin{equation*}
\int_{P} f = f\big(x_0,y_0,z_0\big)
\end{equation*}
More generally if, for each $1\le i\le \ell$, $P_i=(x_i,y_i,z_i)\in\bbbr^3$
is a point and $n_i$ is an integer, then
\begin{equation*}
\int_{\Sigma_{i=1}^\ell n_iP_i} f = \sum_{i=1}^\ell n_i f\big(x_i,y_i,z_i\big)
\end{equation*}
\item
Let  $\om = \vF(\vr)\cdot\dee{\vr}
 = F_1(x,y,z)\,\dee{x} + F_2(x,y,z)\,\dee{y} + F_3(x,y,z)\,\dee{z} $ 
be a  $1$-form. Let $\cC$ be a curve that is parametrized 
by $\vr(t) = \big(x(t)\,,\,y(t)\,,\,z(t)\big)$, $a\le t\le b$. Then,
motivated by Example \ref{eg:diffFormDiffB}.a above,
\begin{equation*}
\int_{\cC}\om = \int_a^b \vF\big(\vr(t)\big)\cdot \diff{\vr}{t}(t)\ \dee{t}
              =\int_{\cC} \vF\cdot\dee{\vr}
\end{equation*}

\end{enumerate}
\end{defn}

\addtocounter{theorem}{-1}
\begin{defn}[continued]
\begin{enumerate}[(a)]


\item[(c)]
Let  $\om = F_1(x,y,z)\,\dee{y}\wedge\dee{z}
 + F_2(x,y,z)\,\dee{z}\wedge\dee{x}
 + F_3(x,y,z)\,\dee{x}\wedge\dee{y}$   
be a  $2$-form. Let $S$ be an oriented surface that is parametrized 
by $\vr(u,v) = \big(x(u,v)\,,\,y(u,v)\,,\,z(u,v)\big)$, 
with $(u,v)$ running over a region $R$ in the $uv$-plane. 
Assume that $\vr(u,v)$ is orientation preserving in the sense that
$\hn\,\dee{S} = +\frac{\partial \vr}{\partial u}
\times \frac{\partial \vr}{\partial v}\,\dee{u}\,\dee{v}$.Then,
motivated by Example \ref{eg:diffFormDiffB}.b above,
\begin{align*}
\int_{S}\om &= \dblInt_R \Big[\vF\big(\vr(u,v)\big)\cdot 
           \frac{\partial\vr}{\partial u}(u,v)
                 \times \frac{\partial\vr}{\partial v}(u,v)\Big] 
         \dee{u}\wedge\dee{v}
           = \dblInt_S \vF\cdot \hn\,\dee{S} 
\end{align*}
%\begin{align*}
%\int_{S}\om &= \dblInt_R \Big[F_1\big(\vr(u,v)\big)\,
%    \Big(\frac{\partial y}{\partial u}\frac{\partial z}{\partial v}
%        -\frac{\partial y}{\partial v}\frac{\partial z}{\partial u}\Big)
%      +F_2\big(\vr(u,v)\big)\,
%    \Big(\frac{\partial z}{\partial u}\frac{\partial x}{\partial v}
%        -\frac{\partial z}{\partial v}\frac{\partial x}{\partial u}\Big)
%   \\&\hskip1in
%      +F_3\big(\vr(u,v)\big)\,
%    \Big(\frac{\partial x}{\partial u}\frac{\partial y}{\partial v}
%        -\frac{\partial x}{\partial v}\frac{\partial y}{\partial u}\Big)
%        \Big] \dee{u}\dee{v}
%\end{align*}


\item[(d)]
Let  $\om = f(x,y,z)\,\dee{x}\wedge\dee{y}\wedge\dee{z}$   
be a  $3$-form. Let $V$ be a solid in $\bbbr^3$. Then
\begin{align*}
\int_{V}\om &= \tripInt_V f(x,y,z)\,\dee{x}\dee{y}\dee{z}
\end{align*}
\end{enumerate}
\end{defn}

Finally, after all of these definitions, we have a very compact
theorem that simultaneously covers the fundamental theorem of calculus, 
Green's theorem. Stokes' theorem and the divergence theorem. Had we given 
all of our definitions in $n$ dimensions, rather than just three dimensions,
it would cover a lot more. This general theorem is also called Stokes'
theorem.
\begin{theorem}[Stokes' Theorem]\label{thm:GenStokes}
If $\om$ is a $k$-form (with $k=0,1,2$) and $D$ is a $(k+1)$-dimensional
domain of integration, then
\begin{equation*}
\int_D d\om=\int_{\partial D}\om
\end{equation*}
Here $\partial D$ is the boundary of $D$ (suitably oriented).
\end{theorem}

To see the connection between the general Stokes' Theorem \ref{thm:GenStokes}
and the Stokes' and divergence theorems of the earlier part of this chapter, 
here are the $k=1$ and $k=2$ cases of Theorem \ref{thm:GenStokes} again.
\begin{itemize}
\item 
  Let $\om = F_1 \dee{x} + F_2 \dee{y} + F_3 \dee{z}$ be a $1$-form
and let $S$ be a piecewise smooth oriented surface as in (our original)
Stokes' theorem \ref{thm:Stokes}. Then, by Lemma \ref{lemma:dGradCurlDiv}.b,
\begin{equation*}
d\om = (\vnabla\times\vF)_1\,\dee{y}\wedge\dee{z}
   +(\vnabla\times\vF)_2\,\dee{z}\wedge\dee{x}
   +(\vnabla\times\vF)_3\,\dee{x}\wedge\dee{y}
\end{equation*}
So, by parts (c) (but with $\vF$ replaced by $\vnabla\times\vF$) and (b)
of Definition \ref{def:differentialFormInt},
the conclusion $\int_D d\om=\int_{\partial D}\om$ of 
(the general) Stokes' Theorem \ref{thm:GenStokes} is
\begin{align*}
\dblInt_S \vnabla\times\vF\cdot\hn\,\dee{S}
=\int_S d\om=\int_{\partial S}\om
=\int_{\partial S}\vF\cdot\dee{\vr}
\end{align*}
which is the conclusion of (our original) Stokes' theorem \ref{thm:Stokes}.

\item 
  $\om = F_1(x,y,z)\,\dee{y}\wedge\dee{z}
 + F_2(x,y,z)\,\dee{z}\wedge\dee{x}
 + F_3(x,y,z)\,\dee{x}\wedge\dee{y}$   
be a  $2$-form
and let $V$ be a solid as in the divergence theorem \ref{thm:divThm}. 
Then, by Lemma \ref{lemma:dGradCurlDiv}.c,
\begin{equation*}
d\om = \vnabla\cdot\vF\,\dee{x}\wedge\dee{y}\wedge\dee{z}
\end{equation*}
So, by parts (d) (with $f =\vnabla\cdot\vF$) and (c)
of Definition \ref{def:differentialFormInt},
the conclusion $\int_D d\om=\int_{\partial D}\om$ of 
(the general) Stokes' Theorem \ref{thm:GenStokes} is
\begin{align*}
\tripInt_V \vnabla\cdot\vF\,\dee{x}\dee{y}\dee{z} 
=\int_V d\om=\int_{\partial V}\om
=\dblInt_{\partial V}\vF\cdot\hn\,\dee{S}
\end{align*}
which is the conclusion of the divergence theorem \ref{thm:divThm}.




\end{itemize}











