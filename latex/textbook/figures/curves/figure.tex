% To create a .eps file, use the command
%      dvips -E -o file.eps figure
% Use gv file.eps to check the bounding box.
% Then to convert the .eps file to a .pdf file while
% preserving the bounding box, use
%         epstopdf  file.eps
% or 
%         ps2pdf -dEPSCrop file.eps

\documentclass[12pt]{article}


\usepackage{amsmath,amsthm,amsfonts, amssymb, mathtools, mathabx}
\usepackage{../clp_macros}
\usepackage{../clp_multi_macrosS}


\thispagestyle{empty}
\input figMac
\def\figdir{}


\begin{document}



\null\vskip1in


%\centerline{\figput{parCurveL}}
%\centerline{\figput{parCircleT}}
%\centerline{\figput{parCurveDerivB}}
%\centerline{\figput{astroid1FF}}
%\centerline{\figput{astroid1A}\quad\figput{astroid1B}\quad\figput{astroid1C}}
%
%\centerline{\figput{astroid1D}\quad\figput{astroid1E}\quad\figput{astroid1F}}
%\centerline{\figput{helix5}}
%   to have a coloured arrowhead in helix3 had to comment out
%              0.0 setgray
%   in one place in the .eps file
%\centerline{\figput{astroid}}
%\centerline{\figput{circle4}}
%\centerline{\figput{corkscrew}}
%\centerline{\figput{reparCircleB}}
%\centerline{\figput{astroidS}}
%\centerline{\figput{curvatureB}}
%\centerline{\figput{curvatureSignD}}
%\centerline{\figput{circleCentreB}}
\centerline{\figput{cross}}
% OBSOLETE
%   to have coloured arrowheads had to comment out
%              0.0 setgray
%   in six places in the .eps file
%\centerline{\figput{wireC}}
%\centerline{\figput{circleC2}}
%\centerline{\figput{skateC}}
%\centerline{\figput{polar}}
%\centerline{\figput{beadRod}}
%\centerline{\figput{beadCurve}}
%\centerline{\figput{equalArea}}
%\centerline{\figput{trianglePl}}
%\centerline{\figput{conic}}
%\centerline{\figput{nocusp}}
%\centerline{\figput{cuspB}}
%\centerline{\figput{helix6}}
%\centerline{\figput{conePlaneCircle}}
%\centerline{\figput{conePlaneParabola}}
%\centerline{\figput{conePlaneHyperbola}}
%\centerline{\figput{conePlaneEllipse}}
%\centerline{\figput{parCircleT}}
%\centerline{\figput{curvatureDef}}




\end{document}